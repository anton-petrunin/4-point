\documentclass[a4paper,10pt]{article}
\usepackage{paper-en}
\usepackage{hyperref}




%\usepackage[notref,notcite,color]{showkeys}



\def\thetitle{Convex cones of metrics}
\def\theauthors{Nina Lebedeva, Anton Petrunin, and Vladimir Zolotov}

\hypersetup{colorlinks=true,
citecolor=black,
linkcolor=black,
anchorcolor=black,
filecolor=black,
menucolor=black,
urlcolor=black,
pdftitle={\thetitle},
pdfauthor={\theauthors}
}








%\usepackage[a-2b,mathxmp]{pdfx}[2018/12/22]
%\overfullrule=100mm
%\usepackage[none]{hyphenat}
\begin{document}
%\pagestyle{empty}\renewcommand\includegraphics[2][{}]{}


\title{\thetitle}
\author{\theauthors}
\date{}
\maketitle

\begin{abstract}
We study the effect on metric spaces imposed by quadratic inequalities on the six distances between points in every quadruple.
\end{abstract}

\section{Quadratic inequality}\label{par:quadratic-inq}

Suppose $a_{i,j}$ are components of a symmetric $n{\times}n$ matrix.
We are interested in length spaces $X$ that
satisfy the inequalities of the form
\[\sum_{i,j}a_{i,j}\cdot|x_i-x_j|_X^2\ge 0\]
for any $n$-point array $\bm{x}\z=(x_1,\dots,x_n)$ in $X$;
here $|x-y|_X$ denotes distance between $x$ and $y$ in $X$.
Inequalities of this kind will be called \emph{quadratic}.

Nonnegative and nonpositive curvature in the sense of Alexandrov can be described by inequalities of that type for $n=4$.
Namely, they are described by the so-called inequalities of negative type $(3,1)$ and $(2,2)$ respectively; see Section~\ref{par:rank-one}.
This observation motivated us to investigate the entire class of such inequalities.
We certainly want to tackle the case $n>4$, but only after the simplest case $n=4$, which is the main subject of this paper.

Our paper contains several simple observations, but altogether they form something interesting and nontrivial; it also lays the groundwork for further investigation.
Sections \ref{Associated form} and \ref{par:rank-one} introduce necessary definitions (associated quadratic form for a point array and inequalities of negative type) and prove several basic statements that valid for all $n$-point arrays.
Further, we turn our attentions to the $4$-point arrays.
In Section~\ref{Four-point arrays}, we prove a new version of a theorem of Abraham Wald \cite[§ 7]{wald} that describes
the metrics of all possible 4-point arrays in Alexandrov spaces with nonnegative and nonpositive curvature.
In Section~\ref{Alexandrov's comparison}, we show that Alexandrov spaces with nonnegative and nonpositive curvature can be defined by arbitrary inequality of negative type $(3,1)$ and $(2,2)$ respectively.
Finally, we show that the globalization theorem holds only for Alexandrov spaces with nonnegative curvature, in other words, for inequalities of negative type $(3,1)$;
see Section~\ref{par:globalization}.


\section{Associated form}\label{Associated form}

Let $\bm{x}=(x_1,\dots,x_n)$ be a point array in a metric space, and let $V_n\z=\RR^{n-1}$.
Choose once and for all a unit-edge regular simplex $\triangle$ in $V_n$; denote by $y_1,\dots,y_n$ its vertices.
Consider the quadratic form $\rho_{\bm{x}}$ on $V_n$ that is defined by the equalities
\[\rho_{\bm{x}}(y_i-y_j)=|x_i-x_j|^2_X\]
for all $i$ and $j$.

The quadratic form $\rho_{\bm{x}}$ will be called the \emph{associated form} of $\bm{x}=(x_1,\z\dots,x_n)$;
it is uniquely defined and remembers all distances $|x_i-x_j|_X$
(we assume that the simplex $\triangle$ in $V_n$ is known).

The Euclidean structure on $V_n$ identifies it with $V_n^*$.
The space of quadratic forms over $V_n$ will be denoted by $W_n$;
it is the symmetric square of $V_n=V_n^*$, and can be written as $W_n=S^2(V_n^*)=S^2(V_n)$.
Any quadratic inequalities described above define linear inequalities on $W_n$, so they can be written as $\langle\omega,\rho_{\bm{x}}\rangle\ge 0$ for a fixed $\omega\in W_n$.

This makes it possible to redefine the spaces that we are going to consider.

Let $K$ be a convex cone in $W_n$;
that is, $K$ is a nonempty set such that if $v,w\in K$, then $a\cdot v+b\cdot w\in K$ for any $a,b\ge0$.
Denote by $\mathcal{M}_K$ the class of all length spaces $X$ such that
$\rho_{\bm{x}}\in K$ for any $n$-point array $\bm{x}\z=(x_1,\dots,x_n)$ in $X$.
The class $\mathcal{M}_K$ respects distance-preserving embeddings; that is,
if there is a distance-preserving embedding $X\to Y$ between length spaces and $Y \in  \mathcal{M}_K$, then $X\in \mathcal{M}_K$.

Suppose $X,Y\in  \mathcal{M}_K$ and $a\ge 0$.
Since $K$ is convex,  $X\times Y$ and $a\cdot X$ are in $\mathcal{M}_K$;
here $X\times Y$ denotes the $\ell_2$-product of metric spaces, and
$a\cdot X$ denotes the rescaled copy of $X$ with factor $a$.

The last observation can be used in the opposite direction as well.
It gives that all forms $\rho_{\bm{x}}$ for point arrays in $\mathcal{M}_K$-spaces form a convex cone, say $K'$.

Evidently $K'\subset K$, and this inclusion might be strict.
First reason comes from the triangle inequality, which is equivalent to $\rho_{\bm{x}}(w)\ge 0$ for any vector $w$ in a 2-face of $\triangle$.
These inequalities must hold for any form in $K'$.
Furthermore, if $\rho_{\bm{x}}\in K'$, then $\rho_{\bm{y}}\in K'$ for any $n$-point array $\bm{y}$ made from the points of $\bm{x}$; in particular $\bm{y}$ might be a permutation of points in $\bm{x}$.

These two conditions hold for all metrics (not necessarily intrinsic),
and one cannot get more for general metric spaces.
The next observation already uses length-metricness.

\begin{thm}{Observation}
If $K$ is a closed convex cone in $W_n$.
Then $K'$ is closed.
Moreover, $\mathcal{M}_K$ is closed under ultralimits.
\end{thm}


(For \emph{ultralimits} and \emph{ultracompletions} of metric spaces and all related topics, see, for example, \cite{petrunin2023}.)

\parit{Proof.}
The last statement is evident, and it implies the first statement.

Indeed, for any sequence of spaces $X_n$ in $\mathcal{M}_K$, its ultralimit $X_\omega$ also belongs to $\mathcal{M}_K$.
Therefore, given a sequence of point arrays $\bm{x}_n$ in $X_n$,
its limit $\bm{x}_\omega$ in $X_\omega$ has limit distances between corresponding points.
Hence the result.
\qeds

\begin{thm}{Proposition}\label{prop:Associated form}
Let $K$ be a closed convex cone in $W_n$ such that $K=K'$.
If $K$ is nontrivial (that is, $K\ne \{0\}$), then $\mathcal{M}_K$ contains all Euclidean spaces.
\end{thm}

\parit{Proof.}
Since $K$ is nontrivial, $\mathcal{M}_K$ contains a space, say $X$, with two distinct points.
Consider the ultracompletion $X^\omega$ of $X$;
recall that $X$ can be (and will be) considered as a subset of $X^\omega$.
Since $K$ is closed, the observation implies that $X^\omega\in \mathcal{M}_K$.

Since  $X$ is a length space, $X^\omega$ has to be geodesic.
Since $X^\omega$ contains a pair of distinct point, it must have a nontrivial geodesic.
It follows that $\mathcal{M}_K$ contains a line segment.
Rescaling the segment and passing to the ultralimit, we get $\RR\in \mathcal{M}_K$;
passing to the products of $\RR$, we get the result. 
\qeds

Let us denote by $Q$ the cone of nonnegative quadratic forms in $W_n$.
Its dual cone $Q^*\subset W_n^*=S^2(V_n)$ is the space generated by tensor squares of vectors in~$V_n$.

\begin{thm}{Corollary}
Let $K$ be a closed convex cone in $W_n$ such that $K=K'$.
Then either $K\supset Q$ or $K$ is trivial; that is, $K=\{0\}$.
\end{thm}

\parit{Proof.}
Since $\RR\in \mathcal{M}_K$, we get that $\sigma^2\in K$ for any linear function $\sigma\:V_n\to\RR$.
By the spectral theorem, any form in $Q$ can be written as a sum of squares of linear functions, hence the result.
\qeds

\section{Rank-one inequalities}\label{par:rank-one}
Recall that $\rho_{\bm{x}}$ denotes the associated quadratic form for a given point array $\bm{x}=(x_1,\dots,x_n)$.

\begin{thm}{Observation}\label{obs:rank-one}
A given point array $\bm{x}=(x_1,\dots,x_n)$ is isometric to an array in a Euclidean space if and only if $\rho_{\bm{x}}(v)\ge 0$ for any vector $v$.
\end{thm}

An inequality of type $\rho_{\bm{x}}(v)\ge 0$ for a fixed vector $v\in V_n$ will be called rank-one inequality.

It is a special case of quadratic inequality.
In the notation of Section~\ref{par:quadratic-inq}, it means that $a_{i,j}=-\lambda_i\cdot\lambda_j$ for a real array $(\lambda_1,\dots, \lambda_n)$ such that
$\lambda_1+\dots+\lambda_n=0$.
Such inequalities are also known as inequalities of \emph{negative type} \cite{deza-lauren}.
If the array $(\lambda_1,\dots, \lambda_n)$ contains $i$ positive and $j$ negative numbers,
then we say that this is an inequality of \emph{negative type} $(i,j)$.
Since changing the signs of all $\lambda_i$ does not change the inequality, we can always assume that $i\ge j$.

\section{Four-point arrays}\label{Four-point arrays}
Now suppose that $n=4$.
In this case, we have two interesting types of rank-one inequalities: negative type $(2,2)$ and $(3,1)$.
The type $(1,1)$ is trivial and
the type $(2,1)$ follows from the triangle inequality.
In fact, the triangle inequality is equivalent to all inequalities of negative type $(2,1)$.

\begin{wrapfigure}{o}{36mm}
\centering
\vskip-4mm
\includegraphics{mppics/pic-20}
\vskip-0mm
\end{wrapfigure}

Consider a rank-one inequality $\rho_{\bm{x}}(v)\ge 0$; we may assume that $v$ is a unit vector.
Furthermore, we may assume that it lies in a closed hemisphere bounded by an equator in the direction of one of the facets of $\triangle$.
These equators divide the hemisphere into triangles and quadrangles.
The inequality $\rho_{\bm{x}}(v)\ge 0$ has negative type $(2,2)$ or $(3,1)$
if and only if $v$ lies in the interior of a triangle or a quadrangle, respectively.

If a vector $v$ is parallel to a facet of $\triangle$, then $\rho_{\bm{x}}(v)\ge 0$ is an inequality of negative type $(2,1)$, which follows from the triangle inequality.
The picture shows the hemisphere.
The labels on the edges show which triangle inequality becomes equality if $\rho_{\bm{x}}$ vanishes at some vector on this edge;
for example, label $123$ means that
\[|x_1-x_2|+|x_2-x_3|=|x_1-x_3|.\]
If $\rho_{\bm{x}}$ vanishes on the intersection of equators, then two points in the array have to coincide;
the label shows which pair.
For example if it is marked by $12$, then $x_1=x_2$.

\begin{thm}{Proposition}\label{prop:Four-point arrays}
Let $X$ be a 4-point metric space.

If $X$ satisfies all inequalities of negative type $(3, 1)$, then it admits an isometric embedding into the product $r\cdot \mathbb{S}^1\times\RR^3$ for some $r>0$.

If $X$ satisfies all inequalities of negative type $(2, 2)$, then it admits an isometric embedding into the product $T\times\RR^3$, where $T$ denotes the \emph{tripod};
that is, three half-lines with a common base point.
\end{thm}

Inequalities of negative type $(3, 1)$ hold in Alexandrov spaces with nonnegative curvature; it follows from the so-called Lang--Schroeder--Sturm inequality \cite{lang-schroeder, sturm}.
Similarly, inequalities of negative type $(2, 2)$ hold in Alexandrov spaces with nonpositive curvature; it follows from \cite[9.5]{AKP-2024}.
Since the tripod $T$ has nonpositive curvature, and $\mathbb{S}^1$ has nonnegative curvature in the sense of Alexandrov, we get the following corollary, which also follows from the result of Abraham Wald \cite[§ 7]{wald}.

\begin{thm}{Corollary}\label{cor:Four-point arrays}
A 4-point metric space $X$ is isometric to subset of a length space with nonpositive (nonnegative) curvature in the sense of Alexandrov if and only if all inequalities of negative type $(3, 1)$ (respectively, type $(2, 2)$) hold in $X$.
\end{thm}

More general five-point versions of this corollary have already been proved by Tetsu Toyoda \cite{toyoda,lebedeva-petrunin2021} and the first two authors \cite{lebedeva-petrunin-2024}, respectively.


\parit{Proof.}
Let us enumerate points in $X$ and let $\rho$ be the associated form on $\RR^3$.
If $\rho\ge 0$, then $X$ is isometric to a 4-point subset of the Euclidean 3-space,
and there is nothing to prove.

Choose a minimal form $\tilde\rho\le \rho$ such that all $(2,1)$-inequalities hold for $\tilde\rho$;
here $\tilde\rho\le \rho$ means that $\tilde\rho(v)\le \rho(v)$ for any vector $v$.
Consider the (semi)metric on $X$ with the associated form $\tilde\rho$;
denote the corresponding metric space by $\tilde X$.
If $X$ satisfies all inequalities of negative type $(3, 1)$ or $(2,2)$, then the same holds for $\tilde X$.

By \ref{obs:rank-one}, $X$ isometrically embeds into $\tilde X\times \RR^3$.
Hence it is sufficient to show that $\tilde X$ embeds into $T$, or, respectively, into $r\cdot \mathbb{S}^1$ for some $r>0$.

Let $N\subset \mathbb{S}^2$ be a set where $\tilde\rho$ is negative and let $\bar N$ be its closure.
The set $N$ is nonempty; otherwise we are in the case $\rho\ge 0$, which is already covered.

Since all $(2,1)$-inequalities hold for $\tilde\rho$,
it must be nonnegative on 4 equators $e_1,e_2,e_3,e_4$ in the directions of facets of $\triangle$.
Since $\tilde\rho$ is minimal, $\bar N$ has to touch $e_i$ at least at 3 directions (up to sign). 
If not, then there is a linear function, say $\sigma$, that vanishes at all common points of $\bar N$ and $e_i$ for all $i$.
In this case, consider the form $\tilde\rho-\eps\cdot \sigma^2$ for small $\eps>0$;
note that all $(2,1)$-inequalities still hold for this form.
Therefore, $\tilde\rho$ is not minimal --- a contradiction.

It means that $\bar N$ lies in a quadrangle or triangle, respectively, and touches its sides at three points or more.

In the case of triangle, $\bar N$ has to touch all of its sides.
The diagram shows that we may, after relabeling, assume that
\begin{align*}
|x_1-x_2|&=|x_1-x_4|+|x_4-x_2|,
\\
|x_2-x_3|&=|x_2-x_4|+|x_4-x_3|,
\\
|x_3-x_1|&=|x_3-x_4|+|x_4-x_1|.
\end{align*}
In this case, the array can be embedded into $T$.

In the case of quadrangle, $\bar N$ touches at least three of its sides (but might touch all four).
The diagram shows that we may, after relabeling, assume that
\begin{align*}
|x_1-x_4|&=|x_1-x_2|+|x_2-x_4|=|x_1-x_3|+|x_3-x_4|,
\\
|x_2-x_3|&=|x_2-x_4|+|x_4-x_3|.
\end{align*}
(If it touches all sides, then in addition we have $|x_2-x_3|=|x_2-x_1|+|x_1-x_3|$.)
In any case, the array can be embedded into $r\cdot \mathbb{S}^1$, where $r=|x_1-x_4|/\pi$.
\qeds

The following picture shows possible positions of the set $N$.
\begin{figure}[h!]
\centering
\vskip-0mm
\includegraphics{mppics/pic-30}
\vskip-0mm
\end{figure}
Below it, we draw a diagram following the convention from \cite{lebedeva-petrunin-2010};
if three points, say $x_1$, $x_2$, and $x_3$, appear in that order on a smooth line, then $|x_1-x_2|+|x_2-x_3|=|x_1-x_3|$.


\section{Alexandrov's comparison}\label{Alexandrov's comparison}

Consider a rank-one inequality $\rho_{\bm{x}}(v)\ge 0$.
This inequality is of negative type $(2,2)$ if $v$ points from one edge of $\triangle$ to the opposite edge.
Further, it is of type $(3,1)$ if $v$ points from a vertex to the opposite facet (up to sign of $v$).

\begin{thm}{Proposition}\label{prop:Alexandrov's comparison}
Suppose $K\subset W_4$ is defined by a single rank-one inequality on four-point arrays.
\begin{enumerate}[(i)]
\item If the inequality is of negative type $(2,2)$, then $\mathcal{M}_K$ consists of all length spaces with nonpositive curvature in the sense of Alexandrov.
\item \label{prop:Alexandrov's comparison:(3,1)} If the inequality is of negative type $(3,1)$, then $\mathcal{M}_K$ consists of all length spaces with nonnegative curvature in the sense of Alexandrov.
\item In the remaining cases, $\mathcal{M}_K$ consists of all length spaces.
\end{enumerate}

\end{thm}

The proof is a straightforward combination of the arguments by Takashi Sato \cite{sato} and its variation by the first two authors \cite{lebedeva-petrunin-2010}.
These papers consider particular cases of $(2,2)$ and $(3,1)$ inequalities, namely for the $\lambda$-arrays $(1,1,-1,-1)$ (type $(2,2)$) and $(1,1,1,-3)$ (type $(3,1)$).

\parit{Proof.}
Our inequality can be written as 
\[\sum_{i,j}\lambda_i\cdot\lambda_j\cdot|x_i-x_j|_X^2\le 0,
\eqlbl{eq:lambda}
\]
where $\lambda_1+\lambda_2+\lambda_3+\lambda_4=0$.
If $\lambda_i=0$ for some $i$,
then the inequality follows from the triangle inequality.
This means that $\mathcal{M}_K$ consists of all length spaces.

It remains to consider the inequalities of negative type $(2,2)$ or $(3,1)$.
In these cases, we can and will assume that our $\lambda$-array is
\[(\alpha\cdot (1-\beta),\  (1-\alpha)\cdot(1-\beta),\  \beta,\ -1)\] 
for some $\alpha,\beta$ such that $0< \beta< 1$.
In case of $(2,2)$-inequality, $1<\alpha$;
in case of $(3,1)$-inequality, $0<\alpha<1$.

Consider the 4-point array $\bm{x}=(x_1,x_2,x_3,x_4)$  in the plane such that 
\[x_4=\alpha\cdot (1-\beta)\cdot x_1+(1-\alpha)\cdot(1-\beta)\cdot x_2+\beta\cdot x_3.\]
We can assume that the array is labeled so that it matches one of the configurations in the pictures below,
so for $(3,1)$-inequality, point $x_4$ lies inside of triangle $x_1x_2x_3$,
and for $(2,2)$-inequality segment $[x_1x_3]$ intersects $[x_2x_4]$.

Note that we get equality in \ref{eq:lambda} for this array.
Moreover, this defines a bijection between $\bm{x}$-arrays up to affine transformation and $\lambda$-arrays up to multiplication by a nonzero coefficient;
as before, we assume that the $\bm{x}$-arrays are in general position --- no three of its points lie on one line.
So we can describe our inequality by a 4-point array in general position.
If the points of the array lie at the vertices of a convex quadrangle,
then it corresponds to an inequality of type $(2,2)$.
If one of the points lies inside of the triangle formed by the remaining points, then it corresponds to an inequality of type $(3,1)$.

\begin{figure}[ht!]
\vskip-0mm
\centering
\includegraphics{mppics/pic-10}
\vskip0mm
\end{figure}

Consider the affine transformation that sends $x_1\mapsto x_1$, $x_2\mapsto x_2$ and $x_3\z\mapsto x_4$;
let $x_5$ be the image of $x_4$ under this transformation
Then
\begin{align*}
x_5&=\alpha\cdot (1-\beta)\cdot x_1+(1-\alpha)\cdot(1-\beta)\cdot x_2+\beta\cdot x_4=
\\
&=\alpha\cdot (1-\beta^2)\cdot x_1+(1-\alpha)\cdot(1-\beta^2)\cdot x_2+\beta^2\cdot x_3.
\end{align*}
Note that $x_4$ lies between $x_3$ and $x_5$;
in particular $|x_3-x_4|+|x_4-x_5|=|x_3-x_5|$.

Consider the inequalities of type \ref{eq:lambda} that correspond to the arrays $x_1,x_2,x_3,x_4$ and $x_1,x_2,x_3,x_5$;
let us call them first and second.

\begin{thm}{Claim}
Suppose that the first inequality holds for any four-point array in a length space $X$, then so does the second inequality.
\end{thm}

Indeed, passing to the ultracompletion, we may assume that $X$ is geodesic.
Choose 4 points $x_1,x_2,x_3,x_5\in X$ and let $x_4$ be a point on a geodesic $[x_3x_5]$ that divides it in the same ratio as in the 5-point configuration in the plane.
If we sum up the first inequality for arrays $x_1,x_2,x_3,x_4$ and $x_1,x_2,x_4,x_5$ with an appropriate coefficients, then we get the second inequality for $x_1,x_2,x_3,x_5$.

By the claim, the inequality for $\lambda$-array $(\alpha\cdot (1-\beta),(1-\alpha)\cdot(1-\beta), \beta,-1)$ implies the inequality for the $\lambda$-array $(\alpha\cdot (1-\beta^2), (1-\alpha)\cdot(1-\beta^2), \beta^2,-1)$.
Applying the claim several times, we get all the inequalities with $\lambda$-arrays
\[(\alpha\cdot (1-\gamma),\  (1-\alpha)\cdot(1-\gamma),\ \gamma,\ -1)\]
for $\gamma=\beta^{2^k}$ with any integer $k\ge 1$;
in particular, we get the following inequality
\[
\begin{aligned}
\alpha\cdot (1-\alpha)\cdot(1-\gamma)^2\cdot|x_1-x_2|^2 - \gamma\cdot |x_3-x_4|^2 &+
\\
+(1-\alpha)\cdot(1-\gamma)\cdot\gamma\cdot|x_2-x_3|^2-\alpha\cdot (1-\gamma)\cdot |x_1-x_4|^2&+
\\
+\alpha\cdot(1-\gamma)\cdot\gamma\cdot|x_1-x_3|^2-(1-\alpha)\cdot(1-\gamma)\cdot |x_2-x_4|^2&\le 0
\end{aligned}
\eqlbl{eq:lambda-inq}
\]
arbitrarily small $\gamma>0$.

Choose $X\in \mathcal{M}_K$.
Let us apply \ref{eq:lambda-inq}
\begin{figure}[ht!]
\vskip-0mm
\centering
\includegraphics{mppics/pic-15}
\vskip0mm
\end{figure}
to a quadruple $x_1,x_2,x_3,x_4\in X$ such that $x_1$, $x_2$ and $x_4$ lie on one geodesic and we have equality in the $(2,1)$-inequality with $\lambda$-array
\[(\alpha,\  (1-\alpha),\ 0,\ -1);\]
that is,
\[\alpha\cdot (1-\alpha)\cdot|x_1-x_2|^2-\alpha\cdot |x_1-x_4|^2-(1-\alpha)\cdot |x_2-x_4|^2=0.\eqlbl{eq:trig-inq}\]
%Then $\tfrac1\gamma\cdot($\ref{eq:lambda-inq}$\,-\,$\ref{eq:trig-inq}$)$ looks like \[\begin{aligned}-\alpha\cdot (1-\alpha)\cdot(2-\gamma)\cdot|x_1-x_2|^2 -  |x_3-x_4|^2 &+\\+(1-\alpha)\cdot(1-\gamma)\cdot|x_2-x_3|^2+\alpha\cdot|x_1-x_4|^2&+\\+\alpha\cdot(1-\gamma)\cdot|x_1-x_3|^2+(1-\alpha)\cdot |x_2-x_4|^2&\le 0.\end{aligned}\]
%Then \ref{eq:trig-inq}$\,+\,\tfrac1\gamma\cdot($\ref{eq:lambda-inq}$\,-\,$\ref{eq:trig-inq}$)$ looks like \[\begin{aligned}-\alpha\cdot (1-\alpha)\cdot(1-\gamma)\cdot|x_1-x_2|^2 -  |x_3-x_4|^2 &+\\+(1-\alpha)\cdot(1-\gamma)\cdot|x_2-x_3|^2&+\\+\alpha\cdot(1-\gamma)\cdot|x_1-x_3|^2&\le 0.\end{aligned}\]
%Since $\gamma>0$ can be taken arbitrary small, we get \[\begin{aligned}-2\cdot \alpha\cdot (1-\alpha)\cdot|x_1-x_2|^2 -  |x_3-x_4|^2 &+\\+(1-\alpha)\cdot|x_2-x_3|^2+\alpha\cdot|x_1-x_4|^2&+\\+\alpha\cdot|x_1-x_3|^2+(1-\alpha)\cdot |x_2-x_4|^2&\le 0.\end{aligned}\] Adding \ref{eq:trig-inq} to the last inequality, we get
Passing to the limit as $\gamma\to 0$ in the inequality \ref{eq:trig-inq}$\,+\,\tfrac1\gamma\cdot($\ref{eq:lambda-inq}$\,-\,$\ref{eq:trig-inq}$)$, we get
\[
\begin{aligned}
\alpha\cdot|x_1-x_3|^2+(1-\alpha)\cdot|x_2-x_3|^2-
\alpha\cdot (1-\alpha)\cdot|x_1-x_2|^2 &\le   |x_3-x_4|^2.\end{aligned}
\eqlbl{eq:CBB-CBA}
\]
%Passing to the limit as $\gamma\to 0$ in the inequality $2\cdot$\ref{eq:trig-inq}$\,+\,\tfrac1\gamma\cdot($\ref{eq:lambda-inq}$\,-\,$\ref{eq:trig-inq}$)$, we get \[ \begin{aligned} |x_3-x_4|^2&\ge (1-\alpha)\cdot|x_3-x_2|^2+\alpha\cdot x_3-x_1|^2- \\ &-(1-\alpha)\cdot |x_4-x_2|^2-\alpha\cdot|x_4-x_1|^2. \end{aligned} \eqlbl{eq:CBB-CBA'} \]
Nonnegative or nonpositive curvature in the sense of Alexandrov can be defined via this inequality for all $\alpha\in (0,1)$ and $\alpha\in (1,\infty)$, respectively; see \cite[8.14 and 9.14]{AKP-2024}.
So far, we obtain it for only one value of $\alpha$.
However, by iterating the inequality, we derive it for the remaining values.

More precisely, assume this inequality holds for some $\alpha\in (0,1)$, then applying iteration we can change $\alpha$ to $(1-\alpha)$, $\alpha^2$,  $\alpha\cdot (1-\alpha)$, $(1-\alpha^2)$, and so on.
Together with any value $\xi$, this set contains $1-\xi$ and $\xi^k$ for any integer $k\ge 1$.
Observe that this set is dense in $(0,1)$.
Therefore, inequality \ref{eq:CBB-CBA} holds for any $\alpha\in (0,1)$.
This is equivalent to the point-on-side comparison for nonnegative curvature; see \cite[8.14]{AKP-2024}.
Similarly one can show that if the inequality holds for some $\alpha>1$, then it holds for any $\alpha>1$,
and this is equivalent to the point-on-side comparison for nonpositive curvature; see \cite[9.14]{AKP-2024}.
\qeds

\section{Globalization}\label{par:globalization}

Let $K\subset W_4$ be a closed convex cone.
We say that a metric space $X$ meets \emph{local $K$-comparison} if any point $x\in X$ admits a neighborhood $U$ such that $K$-comparison holds for any $n$-point array in $U$.

If local $K$-comparison implies $K$-comparison for any length space, then we say that \emph{globalization holds} for $K$.

We continue to consider 4-point arrays, so $V_4=\RR^3$ and $W_4=S^2(V_4^*)=\RR^6$.

\begin{thm}{Theorem}\label{thm:globalization}
Suppose that the globalization holds a closed convex cone $K\z\subset W_4$.
Assume that the $K$-comparison is not trivial;
that is, $\mathcal{M}_K$ does not include all length spaces.
Also, suppose $\mathcal{M}_K$ contains a space with at least two distinct points.
Then $\mathcal{M}_K$ consists of Alexandrov spaces with nonnegative curvature.
\end{thm}

\begin{thm}{Lemma}\label{lem:globalization}
Under the assumptions of the theorem, $\mathcal{M}_K$ contains all Alexandrov spaces with nonnegative curvature.
\end{thm}

\parit{Proof.}
Since $K$-comparison is not trivial, $K\ne\{0\}$.
By \ref{prop:Associated form}, $\mathcal{M}_K$ contains the real line.
Therefore, local $K$-comparison holds
for any circle $r\cdot \mathbb{S}^1$ with $r>0$, and hence also for any product space $r\cdot \mathbb{S}^1\times\RR^3$.
It remains to apply \ref{prop:Four-point arrays}.
\qeds

\parit{Proof of the theorem.}
Let us denote by $K_0$ the cone in $W_4$ described by all inequalities of negative type $(3,1)$ and $(2,1)$.
By \ref{cor:Four-point arrays}, $K_0$ describes all metrics on 4-point arrays in Alexandrov spaces with nonnegative curvature.
By \ref{lem:globalization},  $K$~includes $K_0$.

Given $\delta>0$, consider all metrics with diameter at most $\delta$ on a 4-point array $\{x_1,x_2,x_3,x_4\}$ in an Alexandrov space with curvature at least $-1$.
Denote by $K_\delta\subset W_4$ the minimal closed convex cone that includes all associated forms of these metrics.
Note that $K_0\z\subset K_\delta$;
moreover, $K_0=\bigcap_{\delta>0} K_\delta$.

Every Riemannian manifold has a local curvature bound at each point
In particular, after appropriate rescaling, a small neighborhood of any point of Riemannian manifold has curvature at least $-1$.
Therefore, any compact Riemannian manifold satisfies local $K_\delta$-comparison.

Suppose $K\supset K_\delta$ for some $\delta>0$.
Since the globalization holds for $K$, the class $\mathcal{M}_K$ contains all compact Riemannian manifolds.
Every finite metric graph can be approximated by compact Riemannian manifolds;
to prove it, embed the graph isometrically in the Euclidean space and pass to a boundary of an appropriate
neighborhood.
(A way stronger result is proved by Vedrin Šahović in his thesis \cite{sahovic2009}.)
Any metric on $\{x_1,x_2,x_3,x_4\}$ admits a distance-preserving embedding into a metric graph, so $K$ contains a form near the associated form for any semimetric on $\{x_1,x_2,x_3,x_4\}$.
Since $K$ is closed, it contains forms associated to all metrics on 4-point set;
so $K$ is defined only by the triangle inequalities.
Hence the $K$-comparison is trivial.

From now on, we can assume that for any $\delta>0$ there is a form $\theta\in K_\delta\setminus K$.
By \ref{cor:squared-sides}, we can choose a $(3,1)$-inequality with the $\lambda$-array of this inequality has form $(\lambda_1,\lambda_2,\lambda_2,-1)$ such that
\[\sum_{i,j}\lambda_i\cdot\lambda_j\cdot|x_i-x_j|_X^2
\le
\delta^2\cdot \lambda_1\cdot\lambda_2\cdot\lambda_3\cdot (|x_1-x_2|_X^2+|x_2-x_3|_X^2+|x_3-x_1|_X^2)
\eqlbl{eq:+squares}\]
for any 4-array $(x_1,x_2,x_3,x_4)$ with its form in $K$.

Let $(\lambda_1,\lambda_2,\lambda_2,-1)$ be a partial limit of $\lambda$-arrays of these inequalities;
that is, for some sequence $\delta_n\to 0^+$, we can choose corresponding forms $\theta_n\in W_4$ and vectors $u_n\in V_4$
so that $u_n\to u_\infty$ as $n\to\infty$, and we assume that $(\lambda_1,\lambda_2,\lambda_2,-1)$ is the $\lambda$-array of the inequality $\rho_{\bm{x}}(u_\infty)$.
We have three options:
\begin{enumerate}[(i)]
\item\label{in} $\lambda_1>0$, $\lambda_2>0$, and $\lambda_3>0$;
\item\label{side} $\lambda_i=0$ for one $i$.
\item\label{vertex} $\lambda_i=0$ for two indexes $i$.
\end{enumerate}

In the first case,  $(\lambda_1,\lambda_2,\lambda_2,-1)$ defines an inequality of negative type $(3,1)$.
This inequality holds for any form in $K$.
Therefore, \ref{prop:Alexandrov's comparison}\ref{prop:Alexandrov's comparison:(3,1)} finishes the proof.

In case \ref{side} we can assume that $\lambda_3=0$, so the $\lambda$-array looks like
\[(\alpha,(1-\alpha),0,-1);\]
it defines an inequality of negative type $(2,1)$.
The inequalities for $u_n$ have $\lambda$-array looks like
\[(\alpha_n\cdot(1-\beta_n),(1-\alpha_n)\cdot(1-\beta_n),\beta_n,-1),\]
where $\alpha_n\to\alpha$ and $\beta_n\to 0^+$ as $n\to\infty$.
Applying the same calculations as in the proof of \ref{prop:Alexandrov's comparison} to these inequalities and passing to the limit we get
\[
\alpha\cdot|x_1-x_3|^2+(1-\alpha)\cdot|x_2-x_3|^2-\alpha\cdot (1-\alpha)\cdot|x_1-x_2|^2
\le
|x_3-x_4|^2
\]
if $x_4$ lies on $[x_2x_3]$ and divides it in the ratio $(1-\alpha):\alpha$.
It remains to follow the end of proof of \ref{prop:Alexandrov's comparison}.

In case \ref{vertex} we can assume that $\lambda_1=\lambda_2=0$, so the $\lambda$-array looks like $(0,0,1,-1)$.
Therefore, the inequalities for $u_n$ have $\lambda$-array looks like
\[(\alpha_n\cdot(1-\beta_n),(1-\alpha_n)\cdot(1-\beta_n),\beta_n,-1),\]
where $0<\alpha_n<1$ and $\beta_n\to 1^-$ as $n\to\infty$.
Let us swing the corresponding inequalities defined by \ref{eq:+squares} to reduce this case to the first two.
Namely, we have that the inequality with array
\[(\alpha_n\cdot(1-\beta_n),(1-\alpha_n)\cdot(1-\beta_n),\beta_n,-1)\]
holds with an error
\[\delta_n^2\cdot\alpha_n\cdot(1-\beta_n)^2\cdot(1-\alpha_n)\cdot\beta_n \cdot (|x_1-x_2|_X^2+|x_2-x_3|_X^2+|x_3-x_1|_X^2).\]
Applying the procedure, we get that inequality with array
\[(\alpha_n\cdot(1-\beta_n^{2^k}),(1-\alpha_n)\cdot(1-\beta_n^{2^k}),\beta_n^{2^k},-1)\]
holds with an error
\[10\cdot2^k\cdot\delta_n^2\cdot\alpha_n\cdot(1-\beta_n)^2\cdot(1-\alpha_n)\cdot\beta_n \cdot (|x_1-x_2|_X^2+|x_2-x_3|_X^2+|x_3-x_1|_X^2).\]
Since $\beta_n\approx 1$, we can choose $k=k(n)$ such that $k\to \infty$ and $n\to \infty$ and $\beta_n^k<\gamma$ for some fixed $\gamma<1$ and
\[
10\cdot2^k\cdot\delta_n^2\cdot \alpha_n\cdot(1-\beta_n)^2\cdot(1-\alpha_n)\cdot\beta_n
<
\delta_n^2\cdot \alpha_n\cdot(1-\beta_n^{2^k})^2\cdot(1-\alpha_n)\cdot\beta_n^{2^k}
\]
This brings us to the case \ref{in} or \ref{side}.
\qeds

\section{Auxiliary statements}

\begin{thm}{Lemma}\label{lem:area-bound}
Let $(\lambda_1,\lambda_2,\lambda_3,-1)$ be the $\lambda$-array of an inequality of type $(3,1)$ and
let $(x_1$, $x_2$, $x_3$, $x_4)$ be a 4-point array in the hyperbolic space.

Then
\[\sum_{i,j}\lambda_i\cdot\lambda_j\cdot|x_i-x_j|_X^2\le \lambda_1\cdot\lambda_2\cdot\lambda_3\cdot\tilde a^2,\]
where $\tilde a$ denotes the area of model triangle $\tilde\triangle(x_1x_2x_3)_\EE^2$.
\end{thm}

By Heron's formula,
\begin{align*}
16\cdot \tilde a^2
\quad=\quad &(|x_1-x_2|^2+|x_2-x_3|^2+|x_3-x_1|^2)^2
\\
-2\cdot &(|x_1-x_2|^4+|x_2-x_3|^4+|x_3-x_1|^4).
\end{align*}
In particular, $\tilde a^2$ is defined by a quadratic form on $W_4$.

\parit{Proof.}
Without loss of generality, we may assume that $x_1$, $x_2$, $x_3$ and $x_4$ lie in the hyperbolic plane; moreover $x_4$ belongs to the solid hyperbolic triangle with vertices $x_1$, $x_2$ and $x_3$.

Denote by $a$ the area of the solid hyperbolic triangle with vertices $x_1$, $x_2$ and $x_3$.
By the Kirszbraun theorem,
\[a\le \tilde a.\]
Since the hyperbolic plane has curvature $-1$, we have
\[\pi-\angk{x_1}{x_2}{x_3}_{\HH^2}-\angk{x_2}{x_3}{x_1}_{\HH^2}-\angk{x_3}{x_1}{x_2}_{\HH^2}=a.\]
Applying all this with the comparison, we get
\begin{align*}
0&\le \angk{x_1}{x_2}{x_3}_{\EE^2}-\angk{x_1}{x_2}{x_3}_{\HH^2}\le \tilde a.
\intertext{Since $x_4$ lies in the solid hyperbolic triangle with vertices $x_1$, $x_2$ and $x_3$, we also have $\angk{x_1}{x_2}{x_4}_{\HH^2}+\angk{x_1}{x_4}{x_3}_{\HH^2}=\angk{x_1}{x_2}{x_3}_{\HH^2}$. Hence, the comparison also implies}
0&\le
\angk{x_1}{x_2}{x_4}_{\EE^2}+\angk{x_1}{x_4}{x_3}_{\EE^2}-\angk{x_1}{x_2}{x_3}_{\HH^2}\le \tilde a.
\end{align*}

Set $\phi=\angk{x_1}{x_2}{x_3}_{\EE^2}$ and $\psi=\angk{x_1}{x_2}{x_4}_{\EE^2}+\angk{x_1}{x_4}{x_3}_{\EE^2}$.
From the above, we get $\phi\le \psi+a$.
By the law of cosines,
\[|x_2-x_3|^2=|x_1-x_2|^2+ |x_1-x_3|^2-2|x_1-x_2|\cdot|x_1-x_3|\cdot\cos\phi\]
Redefining the distance $|x_2-x_3|$ via the law of cosines with angle $\psi$,
\[|x_2-x_3|^2\mathrel{:=}|x_1-x_2|^2+ |x_1-x_3|^2-2|x_1-x_2|\cdot|x_1-x_3|\cdot\cos\psi\]
yields a Euclidean quadruple.
That is, decreasing $|x_2-x_3|^2$ by
\[2\cdot |x_1-x_2|\cdot|x_1-x_3|(\cos\phi-\cos\psi)
\le
2\cdot|x_1-x_2|\cdot|x_1-x_3|\cdot a\cdot\sin\phi
=
4\cdot a^2\]
makes the quadruple Euclidean.
It follows that
\[\sum_{i,j}\lambda_i\cdot\lambda_j\cdot|x_i-x_j|_X^2\le 4\cdot \lambda_2\cdot\lambda_3\cdot\tilde a^2,\]
We may assume that $\lambda_1\ge \lambda_2\ge \lambda_3>0$;
therefore, $\lambda_1\ge \tfrac13$, and the statement follows.
\qeds

Recall that $K_0$ denotes the cone in $W_4$ described by all inequalities of negative type $(3,1)$ and $(2,1)$.

\begin{thm}{Corollary}\label{cor:squared-sides}
Let $K\subset W_4$ be a convex closed cone.
Suppose that $K\supset K_0$, but $K$-comparison does not hold for a 4-array of diameter at most $\eps$ in the hyperbolic space.
Then there is an inequality of negative type $(3,1)$ with a $\lambda$-array $(\lambda_1,\lambda_2,\lambda_3,-1)$ such that the inequality
\[\sum_{i,j}\lambda_i\cdot\lambda_j\cdot|x_i-x_j|_X^2\le \eps^2\cdot \lambda_1\cdot\lambda_2\cdot\lambda_3\cdot (|x_1-x_2|_X^2+|x_2-x_3|_X^2+|x_3-x_1|_X^2)\]
holds for any quadruple in $K$.
\end{thm}

\parit{Proof.}
Let $\theta$ be the associated form of the 4-array in the hyperbolic space.
Since $K$ is convex, we can choose a quadratic inequality that does not hold for $\theta$, but holds for any form in $K$.
Our inequality can be written as $\langle \omega,\rho_{\bm{x}} \rangle\ge 0$, where $\rho_{\bm{x}}\in W_4$ is an associated form of some 4-point array $(x_1,x_2,x_3,x_4)$ and $\omega$ is a fixed unit form in $W_4$.

(Recall that $V_4$ and $W_4$ come with natural scalar product; see Section~\ref{Associated form}.
In particular, we have identification $V_4=V_4^*$ and $W_4=S^2(V_4^*)=S^2(V_4)=W_4^*$.
The scalar product on $W_4$ satisfies the identity $\langle \sigma^2,\tau^2\rangle=\langle \sigma,\tau\rangle^2$ for $\sigma,\tau\in V_4^*$.)

\begin{wrapfigure}{o}{30mm}
\centering
\vskip-0mm
\includegraphics{mppics/pic-40}
\vskip-0mm
\end{wrapfigure}

Let $N\subset \SSS^2\subset V_4$ be the set of unit vectors $v$ such that $\theta(v)<0$.
The set $N$ does not intersect the four equators $e_1$, $e_2$, $e_3$, $e_4$ parallel to the facets of $\triangle$.
Moreover, since $K\supset K_0$, the set $N$ lies in one of triangles, say $\Upsilon$;
we can assume that equators $e_1$, $e_2$, $e_3$ are extensions of sides of $\Upsilon$, and $e_4$ is the remaining equator.

Since $K\supset K_0$, we have $\omega(v)\ge 0$ for any $v\in V_4$.
Let $\sigma$ be a unit 1-form on $V_4$ that vanishes on $e_4$.
By the spectral theorem, $\omega$ and $\sigma^2$ can be diagonalized in common basis (which does not have to be orthogonal).
Therefore,
\[\langle \omega,\rho_{\bm{x}}\rangle=\rho_{\bm{x}}(u)+\rho_{\bm{x}}(v)+\rho_{\bm{x}}(w)\eqlbl{eq:v_123}\]
for fixed vectors $u,v,w\in V_4$ such that $v$ and $w$ point in the direction of the equator $e_4$ or vanish;
the latter condition follows since $\sigma^2$ is also diagonalized.

The inequalities $\rho_{\bm{x}}(v)\ge 0$ and
$\rho_{\bm{x}}(w)\ge 0$ are of type $(2,1)$;
they follow from the triangle inequality, so we always have that $\rho_{\bm{x}}(v)\ge 0$ and
$\rho_{\bm{x}}(w)\ge 0$.
Moreover the values $\rho_{\bm{x}}(v)$ and $\rho_{\bm{x}}(w)$ depend only on sides of triangle $[x_1x_2x_3]$.

Since
\begin{align*}
\langle\omega,\theta\rangle=\theta(u)+\theta(v)+\theta(w)&<0,
&
\theta(v)&\ge0,
&
\theta(w)&\ge0,
\end{align*}
 we have that $u$ points in a direction of $N$.
(In particular, $N\ne\emptyset$.)
Therefore, $\rho_{\bm{x}}(u)\ge0$ is an inequality of negative type $(3,1)$.

Let us rescale $u$, $v$, and $w$ so that the inequality $\theta(u)\ge 0$ (which as we know has negative type $(3,1)$)
has $\lambda$-array $(\lambda_1,\lambda_2,\lambda_3,-1)$.
By \ref{lem:area-bound}, we have
\[\theta(v)+\theta(w)
\le
\lambda_1\cdot\lambda_2\cdot\lambda_3\cdot\tilde a^2.\]
Note that $\theta(v)=\eps_v\cdot d_v^2$ and $\theta(w)=\eps_w\cdot d_w^2$, where
$d_v$ and $d_w$ are distances from a vertex of the model triangle $\tilde\triangle(x_1x_2x_3)$ to a point that divide the opposite side in the certain ratio and $\eps_v,\eps_w\ge 0$,
Since diameter of $\tilde\triangle(x_1x_2x_3)$ is at most $\eps$,
\begin{align*}
\tilde a&\le \eps\cdot d_v/2,
\\
\tilde a&\le \eps\cdot d_w/2.
\end{align*}
Hence $\eps_v+\eps_w\le \eps^2$.
Furthermore, each value $d_v$ and $d_w$ does not exceed the maximal side of $[x_1x_2x_3]$ --- hence the result.
\qeds

\section{Remarks}

The requirement that the cone $K$ is closed in \ref{thm:globalization} is necessary;
without it, the globalization holds for the cone described by inequality
\[|x_1-x_2|^2+|x_1-x_3|^2+|x_2-x_3|^2<4\cdot(|x_1-x_4|^2+|x_2-x_4|^2+|x_3-x_4|^2).\]
To verify this statement, note that the inequality forbids tripods.

It would be super-nice to find a quadratic inequality for $n$-point arrays with globalization that is not implied by the standard globalization theorem.
Some candidates can be found in our earlier paper \cite{lebedeva-petrunin-zolotov}.
Also, it might be possible to prove a version of our globalization theorem
for simply connected spaces, which includes the Cartan--Hadamard theorem

{\sloppy
\def\emph{\textit}
\printbibliography[heading=bibintoc]
\fussy
}

\end{document}
