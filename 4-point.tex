\documentclass[a4paper,10pt]{article}
\usepackage{paper-en}
\usepackage{hyperref}

%\usepackage[notref,notcite,color]{showkeys}

\def\thetitle{Quadratic metric comparisons}
\def\theauthors{Nina Lebedeva, Anton Petrunin, and Vladimir Zolotov}

\hypersetup{colorlinks=true,
citecolor=black,
linkcolor=black,
anchorcolor=black,
filecolor=black,
menucolor=black,
urlcolor=black,
pdftitle={\thetitle},
pdfauthor={\theauthors}
}

%\usepackage[a-2b,mathxmp]{pdfx}[2018/12/22]
%\overfullrule=100mm
%\usepackage[none]{hyphenat}
\begin{document}
%\pagestyle{empty}\renewcommand\includegraphics[2][{}]{}

\title{\thetitle}
\author{\theauthors}
\date{}
\maketitle

\begin{abstract}
We study the effect on length-metric spaces imposed by quadratic inequalities on the six distances between points in every quadruple.
\end{abstract}

\section{Introduction}\label{par:quadratic-inq}

\paragraph{Quadratic condition.}
Let $\bm{x}\z=(x_1,\dots,x_n)$ be an $n$-point array in a metric space $X$ and let $a_{i,j}$ be components of a symmetric $n{\times}n$ matrix.
An inequality of the following type
\[\sum_{i,j}a_{i,j}\cdot|x_i-x_j|_X^2\ge 0\]
will be called \emph{quadratic}.
A system of quadratic inequalities will be called \emph{quadratic condition}.

We will be interested in length spaces $X$ such that a given quadratic condition holds for any $n$-point array in $X$.
The following statement says that the Toponogov theorem has no simple relatives.
It will be proved in Section~\ref{par:globalization}.

\begin{thm}{Main theorem}
Suppose that a quadratic condition on 4-points satisfies the globalization property;
that is, if this condition holds locally (in a neighborhood of any point) in a length space $X$, then it holds in $X$.
Then either the condition is trivial, or it describes Alexandrov spaces with nonnegative curvature.
\end{thm}


\paragraph{Auxiliary resuls.}
Let us state two auxiliary results that have independent interest.
Their formulations use a special case of quadratic inequalities
with $a_{i,j}=-\lambda_i\cdot\lambda_j$ for a real array $(\lambda_1,\dots, \lambda_n)$ such that $\lambda_1+\dots+\lambda_n=0$; they are discussed in Section~\ref{par:rank-one}.

In Section~\ref{Four-point arrays}, we prove the following version of a theorem of Abraham Wald \cite[§ 7]{wald} that describes the metrics of all possible 4-point arrays in Alexandrov spaces with nonnegative and nonpositive curvature.

\begin{thm}{Proposition}
The following three conditions are equivalent:
\begin{enumerate}[(i)]
\item A 4-point metric space $X$ is isometric to a subset of a length space with nonnegative (nonpositive) curvature in the sense of Alexandrov.
\item $X$ is isometric to a subset in the product of $\RR^3\times r\cdot \SSS^1$ for some $r>0$ (respectively, $\RR^3\times Y$, where $Y$ denotes the tripod; that is, three half-lines with a common base point)
\item All inequalities of negative type with $\lambda$-arrays such that $\lambda_1\cdot\lambda_2\cdot\lambda_3\cdot\lambda_4<0$ (respectively, type $\lambda_1\cdot\lambda_2\cdot\lambda_3\cdot\lambda_4>0$) hold in $X$.
\end{enumerate}
\end{thm}

Further, in Section~\ref{Alexandrov's comparison}, we prove that a single nondegenerate inequality of negative type defines Alexandrov spaces with nonnegative and nonpositive curvature.

\begin{thm}{Proposition}
Let us consider all length spaces that satisfy the quadratic condition given by an inequality of negative type $\lambda$-array $(\lambda_1,\lambda_2,\lambda_3,\lambda_4)$.
\begin{enumerate}[(i)]
\item If $\lambda_1\cdot\lambda_2\cdot\lambda_3\cdot\lambda_4<0$, then it describes all Alexandrov spaces with nonnegative curvature.
\item If $\lambda_1\cdot\lambda_2\cdot\lambda_3\cdot\lambda_4>0$, then it describes all Alexandrov spaces with nonpositive curvature.
\item If $\lambda_1\cdot\lambda_2\cdot\lambda_3\cdot\lambda_4=0$, then it gives no restrictions; it describes all length spaces.
\end{enumerate}

\end{thm}

This statement shows that a relatively weak form of 4-point comparison forces a much stronger 4-point comparison (once we assume that the metric is intrinsic).
The proof is a straightforward combination of the arguments by Takashi Sato \cite{sato} and its variation by the first two authors \cite{lebedeva-petrunin-2010};
these papers consider particular cases of such inequalities, namely those for the $\lambda$-arrays $(1,1,-1,-1)$ and $(1,1,1,-3)$.

Sections \ref{Associated form} and \ref{par:rank-one} introduce necessary definitions.
Here we define the associated quadratic form for a point array and inequalities of negative type.
We also prove several basic statements that are valid for all $n$-point arrays.
Section \ref{Auxiliary statements} provides a technical statement for the proof of the main theorem.

\section{Associated form}\label{Associated form}

Choose a point array $\bm{x}=(x_1,\dots,x_n)$ in a metric space.
Let $V_n\z=\RR^{n-1}$ be Euclidean space with a choice of unit-edge regular simplex $\triangle$, whose vertices are $y_1,\dots,y_n$.
Consider the quadratic form $\rho_{\bm{x}}$ on $V_n$ that is defined by the equalities
\[\rho_{\bm{x}}(y_i-y_j)=|x_i-x_j|^2_X\]
for all indices $i$ and $j$.

The quadratic form $\rho_{\bm{x}}$ will be called the \emph{associated form} of $\bm{x}=(x_1,\z\dots,x_n)$;
it is uniquely defined and remembers all distances $|x_i-x_j|_X$
(we assume that the simplex $\triangle$ in $V_n$ is known).
We may not distinguish between $\rho_{\bm{x}}$ and the corresponding semimetric on $\bm{x}=(x_1,\dots,x_n)$.

The Euclidean structure on $V_n$ identifies it with $V_n^*$.
The space of quadratic forms over $V_n$ will be denoted by $W_n$;
it is the symmetric square of $V_n=V_n^*$, and can be written as $W_n=S^2(V_n^*)=S^2(V_n)$.

Any quadratic inequality described above defines linear inequalities on $W_n$, so they can be written as $\langle\omega,\rho_{\bm{x}}\rangle\ge 0$ for a fixed $\omega\in W_n$.
Therefore, any quadratic  condition defines a closed convex cone in $W_n$, say $K$;
that is, $K$ is a nonempty closed set such that if $v,w\in K$, then $a\cdot v+b\cdot w\in K$ for any $a,b\ge0$.
Denote by $\mathcal{M}_K$ the class of all length spaces $X$, such that
$\rho_{\bm{x}}\in K$ for any $n$-point array $\bm{x}\z=(x_1,\dots,x_n)$ in~$X$.
The class $\mathcal{M}_K$ respects distance-preserving embeddings; that is,
if there is a distance-preserving embedding $X\to Y$ between length spaces and $Y \in  \mathcal{M}_K$, then $X\in \mathcal{M}_K$.

The following two observations were made by Alexandr Andoni, Assaf Naor, and Ofer Neiman \cite[1.4.1]{ANN}.

Since $K$ is a convex cone,
\[X,\ Y\in  \mathcal{M}_K
\qquad\Longrightarrow\qquad
X\times Y,\ a\cdot X\in\mathcal{M}_K
\]
for any $a\ge 0$;
here $X\times Y$ denotes the $\ell_2$-product of metric spaces, and
$a\cdot X$ denotes the rescaled copy of $X$ with factor $a$.
Moreover, this can be used in the opposite direction as well.
It gives that all forms $\rho_{\bm{x}}$ for point arrays $\bm{x}$ in $\mathcal{M}_K$-spaces form a convex cone.
Let us denote this new cone by $K'$.

Evidently $K'\subset K$, and this inclusion might be strict.
The first reason comes from the triangle inequality, which is equivalent to $\rho_{\bm{x}}(w)\ge 0$ for any vector $w$ in 2-faces of $\triangle$.
These inequalities must hold for any form in $K'$.
Furthermore, if $\rho_{\bm{x}}\in K'$, then $\rho_{\hat{\bm{x}}}\in K'$ for any $n$-point array $\hat{\bm{x}}$ made from the points of $\bm{x}$; in particular $\hat{\bm{x}}$ might be a permutation of points in $\bm{x}$.

These two conditions hold for all metrics (not necessarily intrinsic),
and one cannot get more for general metric spaces.
The next observation already uses length-metricness.

\begin{thm}{Observation}
If $K$ is a closed convex cone in $W_n$, then $K'$ is closed.
Moreover, $\mathcal{M}_K$ is closed under ultralimits.
\end{thm}

(For \emph{ultralimits} and \emph{ultracompletions} of metric spaces and all related topics, see, for example, \cite{petrunin2023}.)

\parit{Proof.}
The last statement is evident, and it implies the first statement.

Indeed, for any sequence of spaces $X_n$ in $\mathcal{M}_K$, its ultralimit $X_\omega$ also belongs to $\mathcal{M}_K$.
Therefore, given a sequence of point arrays $\bm{x}_n$ in $X_n$,
its ultralimit $\bm{x}_\omega$ in $X_\omega$ has limit distances between corresponding points.
Hence, the result.
\qeds

\begin{thm}{Proposition}\label{prop:Associated form}
Let $K$ be a closed convex cone in $W_n$ such that $K=K'$.
If $K$ is nontrivial (that is, $K\ne \{0\}$), then $\mathcal{M}_K$ contains all Euclidean spaces.
\end{thm}

\parit{Proof.}
Since $K$ is nontrivial, $\mathcal{M}_K$ contains a space, say $X$, with two distinct points.
Consider the ultracompletion $X^\omega$ of $X$;
the space $X$ will be considered as a subset of $X^\omega$.
Since $K$ is closed, the observation implies that $X^\omega\in \mathcal{M}_K$.

Since  $X$ is a length space, $X^\omega$ has to be geodesic.
Since $X^\omega$
contains a pair of distinct points, it must contain a nontrivial geodesic.
It follows that $\mathcal{M}_K$ contains a line segment.
By rescaling the segment and passing to the ultralimit, we get $\RR\in \mathcal{M}_K$;
taking products of real lines then yields the result.
\qeds

Let us denote by $Q$ the cone of nonnegative quadratic forms in $W_n$.
Its dual cone $Q^*\subset W_n^*=S^2(V_n)$ is generated by tensor squares of vectors in~$V_n$.

\begin{thm}{Corollary}
Let $K$ be a closed convex cone in $W_n$ such that $K=K'$.
Then either $K\supset Q$ or $K$ is trivial; that is, $K=\{0\}$.
\end{thm}

\parit{Proof.}
Since $\RR\in \mathcal{M}_K$, we get that $\sigma^2\in K$ for any linear function $\sigma\:V_n\to\RR$.
By the spectral theorem, any form in $Q$ can be written as a sum of squares of linear functions, hence the result.
\qeds

\section{Rank-one inequalities}\label{par:rank-one}
Recall that $\rho_{\bm{x}}$ denotes the associated quadratic form for a given point array $\bm{x}=(x_1,\dots,x_n)$.

\begin{thm}{Observation}\label{obs:rank-one}
A point array $\bm{x}=(x_1,\dots,x_n)$ is isometric to an array in a Euclidean space if and only if $\rho_{\bm{x}}(v)\ge 0$ for any vector $v$.
\end{thm}

An inequality of type $\rho_{\bm{x}}(v)\ge 0$ for a fixed vector $v\in V_n$ will be called \emph{rank-one} inequality.
It belongs to the class of quadratic inequalities.  
In the notation of Section~\ref{par:quadratic-inq}, it means that $a_{i,j}=-\lambda_i\cdot\lambda_j$ for a real array $(\lambda_1,\dots, \lambda_n)$ such that
$\lambda_1+\dots+\lambda_n=0$.
Such inequalities are also known as inequalities of \emph{negative type} \cite{deza-lauren}.
If the array $(\lambda_1,\dots, \lambda_n)$ contains $i$ positive and $j$ negative numbers,
then we say that this is an inequality of \emph{negative type} $(i,j)$.
Since changing the signs of all $\lambda_i$ does not change the inequality, we can always assume that $i\ge j$.

\section{Four-point arrays}\label{Four-point arrays}

For $4$-point arrays, we have two interesting types of rank-one inequalities: negative type $(2,2)$ and $(3,1)$.
The type $(1,1)$ is trivial, and
the type $(2,1)$ follows from the triangle inequality.
In fact, all the triangle inequalities (we have 12 of them for 4 points) are equivalent to all inequalities of negative type $(2,1)$ (there are infinitely many of them).

Consider a rank-one inequality $\rho_{\bm{x}}(v)\ge 0$; we may assume that $v$ is a unit vector.
Since the sign of $v$ does not change the inequality, we may assume that it lies in a closed hemisphere bounded by an equator in the direction of one of the facets of $\triangle$.
These equators divide the hemisphere into 4 triangles and 3 quadrangles.
The inequality $\rho_{\bm{x}}(v)\ge 0$ has negative type $(2,2)$ or $(3,1)$
if and only if $v$ lies in the interior of a triangle or a quadrangle, respectively.
Equivalently, this inequality is of negative type $(2,2)$ if $v$ points from one edge of the tetrahedron $\triangle$ to the opposite edge, and of type $(3,1)$ if $v$ points from a vertex to the opposite facet (up to the sign of $v$).

\begin{wrapfigure}{o}{36mm}
\centering
\vskip-3mm
\includegraphics{mppics/pic-20}
\vskip-0mm
\end{wrapfigure}

If a vector $v$ is parallel to a facet of $\triangle$, then $\rho_{\bm{x}}(v)\ge 0$ is an inequality of negative type $(2,1)$, which follows from the triangle inequality.
The picture shows the hemisphere.
The labels on the edges indicate which triangle inequality becomes an equality when $\rho_{\bm{x}}$ vanishes at a vector on that edge;
for example, the label $123$ means that
\[|x_1-x_2|+|x_2-x_3|=|x_1-x_3|.\]
If $\rho_{\bm{x}}$ vanishes on the intersection of equators, then two points in the array have to coincide;
the label shows which pair.
For example, if it is marked by $12$, then $x_1=x_2$.

\begin{thm}{Proposition}\label{prop:Four-point arrays}
Let $X$ be a 4-point metric space.

$X$ satisfies all inequalities of negative type $(3, 1)$, if and only if it admits an isometric embedding into the product $r\cdot \mathbb{S}^1\times\RR^3$ for some $r>0$.

$X$ satisfies all inequalities of negative type $(2, 2)$, if and only if it admits an isometric embedding into the product $Y\times\RR^3$, where $Y$ denotes the \emph{tripod};
that is, three half-lines with a common base point.
\end{thm}

Inequalities of negative type $(3, 1)$ hold in Alexandrov spaces with nonnegative curvature; it follows from the so-called Lang--Schroeder--Sturm inequality \cite{lang-schroeder, sturm}.
Similarly, inequalities of negative type $(2, 2)$ hold in Alexandrov spaces with nonpositive curvature.
This follows easily from the (2+2)-point comparison \cite[9.5]{AKP-2024}.
Since the tripod $Y$ has nonpositive curvature, and $\mathbb{S}^1$ has nonnegative curvature in the sense of Alexandrov, we get the following corollary, which also follows from the result of Abraham Wald \cite[§ 7]{wald}.

\begin{thm}{Corollary}\label{cor:Four-point arrays}
A 4-point metric space $X$ is isometric to a subset of a length space with nonnegative (nonpositive) curvature in the sense of Alexandrov if and only if all inequalities of negative type $(3, 1)$ (respectively, type $(2, 2)$) hold in $X$.
\end{thm}

The corresponding five-point versions of this corollary have been proved by Tetsu Toyoda \cite{toyoda,lebedeva-petrunin2021} and the first two authors \cite{lebedeva-petrunin-2024}, respectively.

\parit{Proof of \ref{prop:Four-point arrays}.}
Let us enumerate points in $X$ and let $\rho$ be the associated form on~$V_4$.

Choose a minimal form $\tilde\rho\le \rho$ such that all $(2,1)$-inequalities hold for $\tilde\rho$;
here $\tilde\rho\le \rho$ means that $\tilde\rho(v)\le \rho(v)$ for any vector $v$.
Consider the (semi)metric on $X$ with the associated form $\tilde\rho$;
denote the corresponding metric space by $\tilde X$.
If $X$ satisfies all inequalities of negative type $(3, 1)$ or $(2,2)$, then the same holds for $\tilde X$.

By \ref{obs:rank-one}, $X$ isometrically embeds into $\tilde X\times \RR^3$.
Hence, it is sufficient to show that $\tilde X$ embeds into $Y$, or, respectively, into $r\cdot \mathbb{S}^1$ for some $r>0$.

Let $N\subset \mathbb{S}^2$ be a set where $\tilde\rho$ is negative and let $\bar N$ be its closure.
We can assume that $N$ is nonempty; otherwise,
$\rho\ge 0$ and by \ref{obs:rank-one}, the space $X$ is isometric to a 4-point subset of the Euclidean 3-space.

Since all $(2,1)$-inequalities hold for $\tilde\rho$, the form $\tilde\rho$
must be nonnegative on 4 equators $e_1,e_2,e_3,e_4$ in the directions of facets of $\triangle$.
Since $\tilde\rho$ is minimal, $\bar N$ has to touch $e_i$ at least at 3 directions (up to sign). 
If not, then there is a linear function, say $\sigma$, that vanishes at all common points of $\bar N$ and $e_i$ for all $i$.
In this case, consider the form $\tilde\rho-\eps\cdot \sigma^2$ for small $\eps>0$;
note that all $(2,1)$-inequalities still hold for this form.
Therefore, $\tilde\rho$ is not minimal --- a contradiction.

It means that $\bar N$ lies in a quadrangle or triangle, respectively, and touches its sides at three points or more.

In the case of a triangle, $\bar N$ has to touch all of its sides.
According to the diagram above, after relabeling, we can assume that
\begin{align*}
|x_1-x_2|&=|x_1-x_4|+|x_4-x_2|,
\\
|x_2-x_3|&=|x_2-x_4|+|x_4-x_3|,
\\
|x_3-x_1|&=|x_3-x_4|+|x_4-x_1|.
\end{align*}
In this case, the array can be embedded into the tripod $Y$.

In the case of a quadrangle, $\bar N$ touches at least three of its sides, but might touch all four.
Look at the diagram and convince yourself that after relabeling, we may assume that
\begin{align*}
|x_1-x_4|&=|x_1-x_2|+|x_2-x_4|=|x_1-x_3|+|x_3-x_4|,
\\
|x_2-x_3|&=|x_2-x_4|+|x_4-x_3|.
\end{align*}
If it touches all sides, then in addition we have $|x_2-x_3|=|x_2-x_1|+|x_1-x_3|$.
In any case, the array can be embedded into $r\cdot \mathbb{S}^1$, where $r=|x_1-x_4|/\pi$;
so the points $x_1$ and $x_4$ become antipodal in $r\cdot \mathbb{S}^1$.
\qeds

The picture shows the possible positions of the set $N$.
\begin{figure}[h!]
\centering
\vskip-0mm
\includegraphics{mppics/pic-30}
\vskip-0mm
\end{figure}
Below it, we provide a diagram following the convention from \cite{lebedeva-petrunin-2024};
if three points, say $x_1$, $x_2$, and $x_3$, appear in that order on a smooth line, then $|x_1-x_2|+|x_2-x_3|=|x_1-x_3|$.

\section{Alexandrov's comparison}\label{Alexandrov's comparison}

\begin{thm}{Proposition}\label{prop:Alexandrov's comparison}
Suppose $K\subset W_4$ is defined by a single rank-one inequality on $4$-point arrays.
\begin{enumerate}[(i)]
\item If the inequality is of negative type $(2,2)$, then $\mathcal{M}_K$ consists of all length spaces with nonpositive curvature in the sense of Alexandrov.
\item \label{prop:Alexandrov's comparison:(3,1)} If the inequality is of negative type $(3,1)$, then $\mathcal{M}_K$ consists of all length spaces with nonnegative curvature in the sense of Alexandrov.
\item In the remaining cases, $\mathcal{M}_K$ consists of all length spaces.
\end{enumerate}

\end{thm}

This statement and Corollary~\ref{cor:Four-point arrays} imply that a single inequality of negative type $(2,2)$ or $(3,1)$ on a length space implies \emph{all} inequalities of the same type.

\parit{Proof.}
Our inequality can be written as 
\[\sum_{i,j}\lambda_i\cdot\lambda_j\cdot|x_i-x_j|_X^2\le 0,
\eqlbl{eq:lambda}
\]
where $\lambda_1+\lambda_2+\lambda_3+\lambda_4=0$.
If $\lambda_i=0$ for some $i$,
then the inequality follows from the triangle inequality.
In this case, $\mathcal{M}_K$ consists of all length spaces.

It remains to consider the inequalities of negative type $(2,2)$ or $(3,1)$.
In these cases, we can assume that our $\lambda$-array is
\[(\alpha\cdot (1-\beta),\  (1-\alpha)\cdot(1-\beta),\  \beta,\ -1)\] 
for some $\alpha,\beta$ such that $0< \beta< 1$;
in case of $(2,2)$-inequality, we have $1<\alpha$, and in case of $(3,1)$-inequality, we have $0<\alpha<1$.

Consider the 4-point array $\bm{x}=(x_1,x_2,x_3,x_4)$  in the plane such that 
\[x_4=\alpha\cdot (1-\beta)\cdot x_1+(1-\alpha)\cdot(1-\beta)\cdot x_2+\beta\cdot x_3.\]
We can assume that the array matches one of the configurations in the pictures below,
so for $(3,1)$-inequality, the point $x_4$ lies inside the triangle $x_1x_2x_3$,
and for $(2,2)$-inequality, the segment $[x_1x_3]$ intersects $[x_2x_4]$.

Note that we get equality in \ref{eq:lambda} for this array.
Moreover, this defines a bijection between plane $\bm{x}$-arrays up to affine transformation and $\lambda$-arrays up to multiplication by a nonzero coefficient;
as before, we assume that the $\bm{x}$-arrays are in general position --- no three of its points lie on one line.
In other words, we can describe our inequality by a 4-point array in general position up to affine transformation.
If the points of the array lie at the vertices of a convex quadrangle,
then it corresponds to an inequality of type $(2,2)$.
If one of the points lies inside the triangle formed by the remaining points, then it corresponds to an inequality of type $(3,1)$.

\begin{figure}[ht!]
\vskip-0mm
\centering
\includegraphics{mppics/pic-10}
\vskip0mm
\end{figure}

Consider the affine transformation that sends $x_1\mapsto x_1$, $x_2\mapsto x_2$, and $x_3\z\mapsto x_4$;
suppose $x_4\mapsto x_5$.
Then
\begin{align*}
x_5&=\alpha\cdot (1-\beta)\cdot x_1+(1-\alpha)\cdot(1-\beta)\cdot x_2+\beta\cdot x_4=
\\
&=\alpha\cdot (1-\beta^2)\cdot x_1+(1-\alpha)\cdot(1-\beta^2)\cdot x_2+\beta^2\cdot x_3.
\end{align*}
Note that $x_4$ lies between $x_3$ and $x_5$;
in particular $|x_3-x_4|+|x_4-x_5|=|x_3-x_5|$.

\begin{thm}{Claim}\label{clm:1=>2}
Consider the inequalities of type \ref{eq:lambda} that correspond to the arrays $x_1,x_2,x_3,x_4$ and $x_1,x_2,x_3,x_5$;
let us call them first and second.

Suppose that the first inequality holds for any four-point array in a length space $X$.
Then the second inequality also holds.
\end{thm}

Indeed, passing to the ultracompletion, we may assume that $X$ is geodesic.
Choose 4 points $x_1,x_2,x_3,x_5\in X$ and let $x_4$ be a point on a geodesic $[x_3x_5]$ that divides it in the same ratio as in the 5-point configuration in the plane.
If we sum up the first inequality for arrays $x_1,x_2,x_3,x_4$ and $x_1,x_2,x_4,x_5$ with the appropriate coefficients, then we get the second inequality for $x_1,x_2,x_3,x_5$.

{\color{blue}
Namely, $x_4$ divides $[x_3x_5]$ in the ratio $1:\beta$.
Therefore,
\[\beta^2\cdot|x_3-x_5|^2 =\beta^2\cdot (1+\beta)\cdot|x_3-x_4|^2+\beta\cdot(1+\beta)\cdot|x_4-x_5|^2.\]
It follows that
\[L_2(x_1,x_2,x_3,x_5)
=
\beta\cdot(1+\beta)\cdot L_1(x_1,x_2,x_3,x_4)+ (1+\beta)\cdot L_1(x_1,x_2,x_4,x_5),\]
where $L_i(p, q, r, s)$ denotes the left-hand side of the $i^{\text{th}}$ inequality \ref{eq:lambda}, corresponding to the array $p, q, r, s$.
In particular, if $L_1 \le 0$ holds for every point array, then so does $L_2 \le 0$.
}

where $L_i(p,q,r,s)$ denotes the left-hand side in \ref{eq:lambda} the $i^{\text{th}}$ inequality and  the array $p,q,r,s$.
In particular $L_1\le 0$ implies $L_2\le 0$ .

By the claim, the inequality for $\lambda$-array $(\alpha\cdot (1-\beta),(1-\alpha)\cdot(1\z-\beta), \beta,-1)$ implies the inequality for the $\lambda$-array $(\alpha\cdot (1-\beta^2), (1-\alpha)\cdot(1-\beta^2), \beta^2,-1)$.
Applying the claim several times, we get all the inequalities with $\lambda$-arrays
\[(\alpha\cdot (1-\gamma),\  (1-\alpha)\cdot(1-\gamma),\ \gamma,\ -1),\]
where $\gamma=\beta^{2^k}$ for an integer $k\ge 1$;
in particular, we get the following inequality
\[
\begin{aligned}
\alpha\cdot (1-\alpha)\cdot(1-\gamma)^2\cdot|x_1-x_2|^2 - \gamma\cdot |x_3-x_4|^2 &+
\\
+(1-\alpha)\cdot(1-\gamma)\cdot\gamma\cdot|x_2-x_3|^2-\alpha\cdot (1-\gamma)\cdot |x_1-x_4|^2&+
\\
+\alpha\cdot(1-\gamma)\cdot\gamma\cdot|x_1-x_3|^2-(1-\alpha)\cdot(1-\gamma)\cdot |x_2-x_4|^2&\le 0
\end{aligned}
\eqlbl{eq:lambda-inq}
\]
for arbitrarily small $\gamma>0$.

Choose $X\in \mathcal{M}_K$.
Let us apply \ref{eq:lambda-inq}
\begin{figure}[ht!]
\vskip-0mm
\centering
\includegraphics{mppics/pic-15}
\vskip0mm
\end{figure}
to a quadruple $x_1,x_2,x_3,x_4\in X$ such that $x_1$, $x_2$ and $x_4$ lie on one geodesic, and we have equality in the $(2,1)$-inequality with $\lambda$-array
\[(\alpha,\  (1-\alpha),\ 0,\ -1);\]
that is,
\[\alpha\cdot (1-\alpha)\cdot|x_1-x_2|^2-\alpha\cdot |x_1-x_4|^2-(1-\alpha)\cdot |x_2-x_4|^2=0.\eqlbl{eq:trig-inq}\]
%Then $\tfrac1\gamma\cdot($\ref{eq:lambda-inq}$\,-\,$\ref{eq:trig-inq}$)$ looks like \[\begin{aligned}-\alpha\cdot (1-\alpha)\cdot(2-\gamma)\cdot|x_1-x_2|^2 -  |x_3-x_4|^2 &+\\+(1-\alpha)\cdot(1-\gamma)\cdot|x_2-x_3|^2+\alpha\cdot|x_1-x_4|^2&+\\+\alpha\cdot(1-\gamma)\cdot|x_1-x_3|^2+(1-\alpha)\cdot |x_2-x_4|^2&\le 0.\end{aligned}\]
%Then \ref{eq:trig-inq}$\,+\,\tfrac1\gamma\cdot($\ref{eq:lambda-inq}$\,-\,$\ref{eq:trig-inq}$)$ looks like \[\begin{aligned}-\alpha\cdot (1-\alpha)\cdot(1-\gamma)\cdot|x_1-x_2|^2 -  |x_3-x_4|^2 &+\\+(1-\alpha)\cdot(1-\gamma)\cdot|x_2-x_3|^2&+\\+\alpha\cdot(1-\gamma)\cdot|x_1-x_3|^2&\le 0.\end{aligned}\]
%Since $\gamma>0$ can be taken arbitrary small, we get \[\begin{aligned}-2\cdot \alpha\cdot (1-\alpha)\cdot|x_1-x_2|^2 -  |x_3-x_4|^2 &+\\+(1-\alpha)\cdot|x_2-x_3|^2+\alpha\cdot|x_1-x_4|^2&+\\+\alpha\cdot|x_1-x_3|^2+(1-\alpha)\cdot |x_2-x_4|^2&\le 0.\end{aligned}\] Adding \ref{eq:trig-inq} to the last inequality, we get
Passing to the limit as $\gamma\to 0$ in the inequality \ref{eq:trig-inq}$\,+\,\tfrac1\gamma\cdot($\ref{eq:lambda-inq}$\,-\,$\ref{eq:trig-inq}$)$, we get
\[
\begin{aligned}
\alpha\cdot|x_1-x_3|^2+(1-\alpha)\cdot|x_2-x_3|^2-
\alpha\cdot (1-\alpha)\cdot|x_1-x_2|^2 \le |x_3-x_4|^2.\end{aligned}
\eqlbl{eq:CBB-CBA}
\]
%Passing to the limit as $\gamma\to 0$ in the inequality $2\cdot$\ref{eq:trig-inq}$\,+\,\tfrac1\gamma\cdot($\ref{eq:lambda-inq}$\,-\,$\ref{eq:trig-inq}$)$, we get \[ \begin{aligned} |x_3-x_4|^2&\ge (1-\alpha)\cdot|x_3-x_2|^2+\alpha\cdot x_3-x_1|^2- \\ &-(1-\alpha)\cdot |x_4-x_2|^2-\alpha\cdot|x_4-x_1|^2. \end{aligned} \eqlbl{eq:CBB-CBA'} \]

Note that if this inequality holds for all $\alpha\in (0,1)$ (or $\alpha\in (1,\infty)$) then we get
a point-on-side comparison for nonnegative (respectively nonpositive) curvature in the sense of Alexandrov \cite[8.14 and 9.14]{AKP-2024}.
Indeed, for Euclidean space, equality in \ref{eq:CBB-CBA} holds for any $\alpha$.
Therefore, if $\alpha<1$, we get $|x_3-x_4|\ge |\tilde x_3-\tilde x_4|$, where $\tilde x_4$ divides the side $[\tilde x_1\tilde x_2]$ of the model triangle $[\tilde x_1\tilde x_2\tilde x_3]=\tilde\triangle(x_1x_2x_3)$ in the same ratio $(1-\alpha):\alpha$.
Similarly, if $\alpha>1$, we get $|x_3-x_1|\le |\tilde x_3-\tilde x_1|$, where $\tilde x_1$ divides the side $[\tilde x_4\tilde x_2]$ of the model triangle $[\tilde x_4\tilde x_2\tilde x_3]=\tilde\triangle(x_4x_2x_3)$ in the same ratio $(\alpha-1):1$.

So far, we have obtained it for only one value of $\alpha$;
however, by applying iteration, we can derive inequalities for the remaining values $\alpha\in (0,1)$ (respectively, $\alpha\in (1,\infty)$).
More precisely, assume this inequality holds for some $\alpha\in (0,1)$,
then we can change $\alpha$ to $(1-\alpha)$, $\alpha^2$,  $\alpha\cdot (1-\alpha)$, $(1-\alpha^2)$, and so on.
In other words, if $\mathcal{A}\subset(0,1)$ is the set of all values $\alpha$ such that \ref{eq:CBB-CBA} holds, then together with any value $\xi$, it contains $1-\xi$ and $\xi^k$ for any integer $k\ge 1$.
This implies that $\mathcal{A}$ is dense in $(0,1)$, and by continuity, we get $\mathcal{A}=(0,1)$.
That is, inequality \ref{eq:CBB-CBA} holds for any $\alpha\in (0,1)$.
This is equivalent to the point-on-side comparison for nonnegative curvature \cite[8.14]{AKP-2024}.
Similarly, one can show that if the inequality holds for some $\alpha>1$, then it holds for any $\alpha>1$,
and this is equivalent to the point-on-side comparison for nonpositive curvature \cite[9.14]{AKP-2024}.
\qeds

\section{Globalization}\label{par:globalization}

Let $K\subset W_n$ be a closed convex cone.
We say that a metric space $X$ meets \emph{local $K$-comparison} if any point $x\in X$ admits a neighborhood $U$, such that $K$-comparison holds for any $n$-point array in $U$.

If local $K$-comparison implies $K$-comparison for any length space, then we say that \emph{globalization holds} for $K$.

The following statement shows that if globalization holds for quadratic comparison, then it is either trivial or characterizes nonnegatively curved Alexandrov spaces;
so, Toponogov's theorem is the only nontrivial globalization theorem for quadratic conditions on 4-point arrays.

\begin{thm}{Theorem}\label{thm:globalization}
Suppose that the globalization holds for a closed convex cone $K\z\subset W_4$.
Assume that the $K$-comparison is not trivial;
that is, on one side $\mathcal{M}_K$ does not include all length spaces, and, on the other side, $\mathcal{M}_K$ contains a space with at least two distinct points.
Then $\mathcal{M}_K$ consists of Alexandrov spaces with nonnegative curvature.
\end{thm}

\begin{thm}{Lemma}\label{lem:globalization}
Under the assumptions of the theorem, $\mathcal{M}_K$ contains all Alexandrov spaces with nonnegative curvature.
\end{thm}

\parit{Proof.}
Since $K$-comparison is not trivial, $K\ne\{0\}$.
By \ref{prop:Associated form}, $\mathcal{M}_K$ contains the real line.
Therefore, local $K$-comparison holds
for any circle $r\cdot \mathbb{S}^1$ with $r>0$, and hence also for any product space $r\cdot \mathbb{S}^1\times\RR^3$.
It remains to apply \ref{prop:Four-point arrays}.
\qeds

In the following proof, we will use one statement from the next section.

\parit{Proof of the theorem.}
Let us denote by $K_0$ the cone in $W_4$ described by all inequalities of negative type $(3,1)$ and $(2,1)$.
By \ref{cor:Four-point arrays}, $K_0$ describes all metrics on 4-point arrays in Alexandrov spaces with nonnegative curvature.
By \ref{lem:globalization},  $K$~includes $K_0$.

Given $\delta>0$, consider all metrics with diameter at most $\delta$ on a 4-point array $\{x_1,x_2,x_3,x_4\}$ in an Alexandrov space with curvature at least $-1$.
Due to Wald's Theorem (see \cite[Exercise 10.7]{AKP-2024}) we can assume that the $4$-point array
comes from a model $k$-plane (that is, a complete simply connected 2-dimensional Riemannian manifold of constant curvature $k$) with $k \ge -1$.
This implies that the array either (a) is isometric to a $4$-point subset of some sphere or (b) is bi-Lipshitz equivalent with the constant $1 + C\cdot \delta^2$ to a $4$-point subset of a Euclidean plane, where $C > 0$ is some absolute constant.
Denote by $K_\delta\subset W_4$ the minimal closed convex cone that includes all associated forms of these metrics.
Note that $K_0\z\subset K_\delta$.
Furthermore, 
%after rescaling the metric on $\{x_1,x_2,x_3,x_4\}$ with %a factor of $\tfrac1\delta$, the array gets diameter at %most $1$ and is embeddable in an Alexandrov space with curvature at least $-\delta^2$; 
any element of $K_\delta$ is bi-Lipshitz equivalent with the constant $1 + C \cdot\delta^2$ to some element of $K_0$;
therefore, $K_0=\bigcap_{\delta>0} K_\delta$.

Every Riemannian manifold admits a local curvature bound at each point.
In particular, after appropriate rescaling, a small neighborhood of any point of a Riemannian manifold has curvature at least $-1$.
Therefore, any compact Riemannian manifold satisfies local $K_\delta$-comparison.

Suppose $K\supset K_\delta$ for some $\delta>0$.
Since globalization holds for $K$, the class $\mathcal{M}_K$ contains all compact Riemannian manifolds.
\textit{Every finite metric graph can be approximated by compact Riemannian manifolds.}
To prove it,
realize the graph in Euclidean space with smooth edges of the same length and take the boundary of an appropriate neighborhood.
(A way stronger result is proved by Vedrin Šahović in his thesis \cite{sahovic2009}.)
Any metric on $\{x_1,x_2,x_3,x_4\}$ admits a distance-preserving embedding into a metric graph, so $K$ contains a form near the associated form for any semimetric on $\{x_1,x_2,x_3,x_4\}$.
Since $K$ is closed, it contains forms associated to all metrics on 4-point set;
so $K$ is defined only by the triangle inequalities, and the $K$-comparison is trivial.

From now on, we can assume that for any $\delta>0$, there is a form $\theta\in K_\delta\setminus K$.
By \ref{cor:squared-sides}, we can find a $(3,1)$-inequality that holds in $K$ with small error.
Namely, there is a $\lambda$-array $(\lambda_1,\lambda_2,\lambda_3,-1)$ such that $\lambda_i>0$ and
\[\begin{aligned}
\sum_{i,j}&\lambda_i\cdot\lambda_j\cdot|x_i-x_j|_X^2
\le
\\
&\le
10\cdot\delta^2\cdot \lambda_1\cdot\lambda_2\cdot\lambda_3\cdot (|x_1-x_2|_X^2+|x_2-x_3|_X^2+|x_3-x_1|_X^2)
\end{aligned}
\eqlbl{eq:+squares}\]
for any 4-array $(x_1,x_2,x_3,x_4)$ with its form in $K$.

Let $\bm{\lambda}_\infty=(\lambda_1,\lambda_2,\lambda_3,-1)$ be a partial limit of these $\lambda$-arrays;
that is, for some sequence $\delta_n\to 0^+$, we can choose corresponding inequalities of the form \ref{eq:+squares} with $\lambda$-arrays $\bm{\lambda}_n$ such that $\bm{\lambda}_n\to \bm{\lambda}_\infty$ as $n\to \infty$.
We have to deal with three cases:
\begin{enumerate}[(i)]
\item\label{in} $\lambda_1>0$, $\lambda_2>0$, and $\lambda_3>0$;
\item\label{side} $\lambda_i=0$ for one index $i$;
\item\label{vertex} $\lambda_i=0$ for two indices $i$.
\end{enumerate}

\parit{Case \ref{in}.}
Note that $(\lambda_1,\lambda_2,\lambda_3,-1)$ defines an inequality of negative type $(3,1)$.
This inequality holds for any form in $K$.
Therefore, \ref{prop:Alexandrov's comparison}\ref{prop:Alexandrov's comparison:(3,1)} finishes the proof.

\parit{Case \ref{side}.}
We can assume that $\lambda_3=0$, so
\[\bm{\lambda}_\infty=(\alpha,(1-\alpha),0,-1);\]
it defines an inequality of negative type $(2,1)$.
Note that
\[\bm{\lambda}_n=(\alpha_n\cdot(1-\beta_n),(1-\alpha_n)\cdot(1-\beta_n),\beta_n,-1),\]
for some sequences $\alpha_n\to\alpha$ and $\beta_n\to 0^+$ as $n\to\infty$.
A bit below we will show that repeating the proof of \ref{prop:Alexandrov's comparison}, we get
\[
\alpha\cdot|x_1-x_3|^2+(1-\alpha)\cdot|x_2-x_3|^2-\alpha\cdot (1-\alpha)\cdot|x_1-x_2|^2
\le
|x_3-x_4|^2
\]
if $x_4$ lies on $[x_1x_2]$ and divides it in the ratio $(1-\alpha):\alpha$.
After that it remains to follow the end of proof of \ref{prop:Alexandrov's comparison}.

Indeed, suppose $x_4$ lies on $[x_1x_2]$ and divides it in the ratio $(1-\alpha_n):\alpha_n$, then
\[\alpha_n\cdot (1-\alpha_n)\cdot|x_1-x_2|^2-\alpha_n\cdot |x_1-x_4|^2-(1-\alpha_n)\cdot |x_2-x_4|^2=0;\leqno{\text{\ref{eq:trig-inq}}'}\]
this will be used instead of \ref{eq:trig-inq} (page \pageref{eq:trig-inq}).
If we assume $\alpha=\alpha_n$ and $\gamma=\beta_n$ in \ref{eq:lambda-inq}, then instead of zero in the right-hand side, we get the following error term:
\[E_n=10\cdot\delta_n^2\cdot \alpha_n\cdot (1-\alpha_n)\cdot(1-\beta_n)^2\cdot \beta_n\cdot(|x_1-x_2|_X^2+|x_2-x_3|_X^2+|x_3-x_1|_X^2);\]
let us label this inequality by  \ref{eq:lambda-inq}$'$.
The inequality \ref{eq:trig-inq}$'\,+\,\tfrac1{\beta_n}\cdot($\ref{eq:lambda-inq}$'\,-\,$\ref{eq:trig-inq}$')$ implies the following:
\[
\begin{aligned}
\alpha_n\cdot|x_1-x_3|^2+(1-\alpha_n)\cdot|x_2-x_3|^2-
\alpha_n&\cdot (1-\alpha_n)\cdot|x_1-x_2|^2 \le
\\
&\le |x_3-x_4|^2 +  E_n/\beta_n+E_n.
\end{aligned}
\]
It remains to observe that $E_n/\beta_n+E_n\to0$ as $n\to\infty$.

\parit{Case \ref{vertex}.}
We can assume that $\lambda_1=\lambda_2=0$, so $\bm{\lambda}_\infty=(0,0,1,-1)$ and
\[\bm{\lambda}_n=(\alpha_n\cdot(1-\beta_n),(1-\alpha_n)\cdot(1-\beta_n),\beta_n,-1),\]
where $0<\alpha_n<1$ and $\beta_n\to 1^-$ as $n\to\infty$.
Let us modify the corresponding inequalities defined by \ref{eq:+squares} to sweep $\bm{\lambda}_\infty$ out of the corner.
Namely, we have a sequence of $(3,1)$-inequalities with array
\[(\alpha_n\cdot(1-\beta_n),(1-\alpha_n)\cdot(1-\beta_n),\beta_n,-1)\]
that hold with error
\[10\cdot\delta_n^2\cdot\alpha_n\cdot(1-\alpha_n)\cdot\beta_n\cdot(1-\beta_n)^2 \cdot (|x_1-x_2|_X^2+|x_2-x_3|_X^2+|x_3-x_1|_X^2).\]

This sequence of inequalities can be used to produce another sequence that is covered in cases \ref{in} or \ref{side}. 

[REDACTED]

To do this, we fix $n$, and thus fix an inequality which corresponds to it, say $I_n = N_n - E_n \le 0$, where $N_n$ is an inequality of negative type and $E_n$ is the error term. We fix $k = k(n)$ such that $\gamma^2\le \beta_n^{2^k}<\gamma$, say for $\gamma=\tfrac{99}{100}$.
We construct a new inequality $I'_n$ from $I_n$
via a telescoping summation of $2^k$ evaluations of $I_n$
arranged as in \ref{clm:1=>2}.

In the following by the target inequality we mean the inequality of the negative type corresponding to the following $\lambda$-array 
\[(\alpha_n\cdot(1-\beta_n^{2^k}),(1-\alpha_n)\cdot(1-\beta_n^{2^k}),\beta_n^{2^k},-1).\] 
Obviously, $|x_1-x_4|_X^2$ has a coefficient $-\alpha_n\cdot(1-\beta_n^{2^k})$ in it. On the other hand it should be more or less clear that the sum of evaluations of $N_n$ is a multiple of the target inequality. Also, it is easy to compute that $|x_1-x_4|_X^2$ has a coefficient $-\alpha_n\cdot(1-\beta_n)\beta_n^{-(2^k - 1)}$ in it. Thus, we can normalize $I'_n$ by multiplying it by $w(1-\beta_n)^{-1}$, where $w = (1-\beta_n^{2^k})\beta_n^{2^k - 1}$ i.e., after this the  the rescaled sum of evaluations of $N_n$ is exactly equal to the target inequality. We denote this normalized version of $I'_n$ by $I''_n$.


For every of $2^k$ summands of $I'_n$ the error terms can be bounded from above by
\[1000\cdot\delta_n^2\cdot\alpha_n\cdot(1-\alpha_n)\cdot\beta_n\cdot(1-\beta_n)^2 \cdot (|x_1-x_2|_X^2+|x_2-x_3|_X^2+|x_3-x_1|_X^2);\]
That is, the error is most 100 times larger than in the original inequality; It follows that for $I''_n$ the whole error term is not greater that 
\[w \cdot 2^k\cdot1000\cdot\delta_n^2\cdot\alpha_n\cdot(1-\alpha_n)\cdot\beta_n\cdot(1-\beta_n) \cdot (|x_1-x_2|_X^2+|x_2-x_3|_X^2+|x_3-x_1|_X^2) := E''_n.\]

From the way we selected $k$ it follows that 
 $2^{k}(1 - \beta_n) / (1 - \beta_n^{2^k})$ is very close to $1$. Thus, 
\[ E''_n
<
C \cdot \delta_n^2\cdot \alpha_n\cdot(1-\beta_n^{2^k})^2\cdot(1-\alpha_n)\cdot\beta_n^{2^k} \cdot (|x_1-x_2|_X^2+|x_2-x_3|_X^2+|x_3-x_1|_X^2),
\]
for some absolute constant $C > 0$.
It remains to observe that this new sequence of inequalities $\{{I''_n}\}_{n=1}^{\infty}$ is covered in cases \ref{in} or \ref{side}.
\qeds

[ORIGINAL]

To do this, we follow the calculations in \ref{clm:1=>2};
that is, we will use each inequality several times for a sequence of point arrays.
We may assume that the error in the intermediate inequalities is at most
\[1000\cdot\delta_n^2\cdot\alpha_n\cdot(1-\alpha_n)\cdot\beta_n\cdot(1-\beta_n)^2 \cdot (|x_1-x_2|_X^2+|x_2-x_3|_X^2+|x_3-x_1|_X^2);\]
That is, the error is most 100 times larger than in the original inequality; if this is not the case, then by the triangle inequality, the target inequality (stated below) will hold trivially.

The target inequality has an array
\[(\alpha_n\cdot(1-\beta_n^{2^k}),(1-\alpha_n)\cdot(1-\beta_n^{2^k}),\beta_n^{2^k},-1)
\]
and error
\[2^k\cdot1000\cdot\delta_n^2\cdot\alpha_n\cdot(1-\alpha_n)\cdot\beta_n\cdot(1-\beta_n)^2 \cdot (|x_1-x_2|_X^2+|x_2-x_3|_X^2+|x_3-x_1|_X^2).\]
Since $\beta_n\to 1^-$, we can choose $k=k(n)$ such that $k\to \infty$ as $n\to \infty$ and $\gamma^2\le \beta_n^{2^k}<\gamma$, say for $\gamma=\tfrac{99}{100}$ and
\[
2^k\cdot\delta_n^2\cdot \alpha_n\cdot(1-\beta_n)^2\cdot(1-\alpha_n)\cdot\beta_n
<
\delta_n^2\cdot \alpha_n\cdot(1-\beta_n^{2^k})^2\cdot(1-\alpha_n)\cdot\beta_n^{2^k}
\]
It remains to observe that this new sequence of inequalities is covered in cases \ref{in} or \ref{side}.
\qeds



\section{Auxiliary statements}\label{Auxiliary statements}

\begin{thm}{Lemma}\label{lem:area-bound}
Let $(\lambda_1,\lambda_2,\lambda_3,-1)$ be the $\lambda$-array of an inequality of type $(3,1)$ and
let $(x_1$, $x_2$, $x_3$, $x_4)$ be a 4-point array in the hyperbolic space.

Then
\[\sum_{i,j=1}^4\lambda_i\cdot\lambda_j\cdot|x_i-x_j|_X^2
\le
(12+3\cdot\delta^2)\cdot\lambda_1\cdot\lambda_2\cdot\lambda_3\cdot\tilde a^2,\]
where $\delta=\max\{|x_1-x_2|,|x_1-x_3|,|x_2-x_3|\}$
and
$\tilde a$ is the area of the model triangle $\tilde\triangle(x_1x_2x_3)_{\EE^2}$.
\end{thm}

By Heron's formula,
\begin{align*}
16\cdot \tilde a^2
\quad=\quad &(|x_1-x_2|^2+|x_2-x_3|^2+|x_3-x_1|^2)^2
\\
-2\cdot &(|x_1-x_2|^4+|x_2-x_3|^4+|x_3-x_1|^4).
\end{align*}
In particular, $\tilde a^2$ is a quadratic form on $W_4$.

\parit{Proof.}
By the Kirszbraun theorem (see \cite{lang-schroeder,AKP-2011} or \cite[Chapter 10]{AKP-2024}), we may assume that $x_1$, $x_2$, $x_3$ and $x_4$ lie in the hyperbolic plane; moreover, $x_4$ belongs to the solid hyperbolic triangle with vertices $x_1$, $x_2$ and $x_3$.
Denote by $a$ the area of this triangle.
Again, by the Kirszbraun theorem,
\[a\le \tilde a.\]

Since the hyperbolic plane has curvature $-1$, we have
\[\pi-\angk{x_1}{x_2}{x_3}_{\HH^2}-\angk{x_2}{x_3}{x_1}_{\HH^2}-\angk{x_3}{x_1}{x_2}_{\HH^2}=a.\]
These inequalities and the comparison imply that
\begin{align*}
0&\le \angk{x_1}{x_2}{x_3}_{\EE^2}-\angk{x_1}{x_2}{x_3}_{\HH^2}\le \tilde a.
\intertext{Since $x_4$ lies in the solid hyperbolic triangle with vertices $x_1$, $x_2$ and $x_3$, we also have $\angk{x_1}{x_2}{x_4}_{\HH^2}+\angk{x_1}{x_4}{x_3}_{\HH^2}=\angk{x_1}{x_2}{x_3}_{\HH^2}$.
Hence, the comparison also implies}
0&\le
\angk{x_1}{x_2}{x_4}_{\EE^2}+\angk{x_1}{x_4}{x_3}_{\EE^2}-\angk{x_1}{x_2}{x_3}_{\HH^2}\le \tilde a.
\end{align*}

Set $\phi=\angk{x_1}{x_2}{x_3}_{\EE^2}$ and $\psi=\angk{x_1}{x_2}{x_4}_{\EE^2}+\angk{x_1}{x_4}{x_3}_{\EE^2}$.
From the above, we get $\phi\le \psi+\tilde a$.
By the law of cosines,
\[|x_2-x_3|^2=|x_1-x_2|^2+ |x_1-x_3|^2-2|x_1-x_2|\cdot|x_1-x_3|\cdot\cos\phi\]
Redefining the distance $|x_2-x_3|$ via the law of cosines with angle $\psi$,
\[|x_2-x_3|^2\mathrel{:=}|x_1-x_2|^2+ |x_1-x_3|^2-2|x_1-x_2|\cdot|x_1-x_3|\cdot\cos\psi\]
yields a Euclidean quadruple.
That is, decreasing $|x_2-x_3|^2$ by
\[s=2\cdot |x_1-x_2|\cdot|x_1-x_3|\cdot(\cos\psi-\cos\phi)\]
makes the quadruple Euclidean.
Since
\[\cos\psi-\cos\phi=\sin\phi\cdot\sin(\phi-\psi)-2\cdot\cos\phi\cdot (\sin\tfrac{\phi-\psi}2)^2,\] we get
\begin{align*}
s
&\le\
2\cdot|x_1-x_2|\cdot|x_1-x_3|\cdot (\tilde a\cdot\sin\phi+\tfrac12\cdot\tilde a^2)
\le
\\
&\le\ (4+\delta^2)\cdot \tilde a^2
\end{align*}

It follows that
\[\sum_{i,j}\lambda_i\cdot\lambda_j\cdot|x_i-x_j|_X^2\le \lambda_2\cdot\lambda_3\cdot(4+\delta^2)\cdot \tilde a^2,\]
We may assume that $\lambda_1\ge \lambda_2\ge \lambda_3>0$.
Since $\lambda_1+ \lambda_2+ \lambda_3=1$, we have $\lambda_1\ge \tfrac13$, and the statement follows.
\qeds

Recall that $K_0$ denotes the cone in $W_4$ described by all inequalities of negative type $(3,1)$ and $(2,1)$.

\begin{thm}{Corollary}\label{cor:squared-sides}
Let $K\subset W_4$ be a convex closed cone.
Suppose that $K\supset K_0$, but $K$-comparison does not hold for a 4-array of diameter at most $\delta$ in the hyperbolic space.
If $\delta$ is sufficiently small,
then there is an inequality of negative type $(3,1)$ with a $\lambda$-array $(\lambda_1,\lambda_2,\lambda_3,-1)$ such that the inequality
\[\sum_{i,j=1}^4\lambda_i\cdot\lambda_j\cdot|x_i-x_j|_X^2
\le
10\cdot\delta^2\cdot \lambda_1\cdot\lambda_2\cdot\lambda_3\cdot (|x_1-x_2|_X^2+|x_2-x_3|_X^2+|x_3-x_1|_X^2)\]
holds for any quadruple in $K$.
\end{thm}

This inequality should be interpreted as an $(3,1)$-inequality with a $\lambda$-array $(\lambda_1,\lambda_2,\lambda_3,-1)$ with error
\[10\cdot\delta^2\cdot \lambda_1\cdot\lambda_2\cdot\lambda_3\cdot (|x_1-x_2|_X^2+|x_2-x_3|_X^2+|x_3-x_1|_X^2).\]

\parit{Proof.}
Let $\theta$ be the associated form of the 4-array in the hyperbolic space.
Since $K$ is convex, we can choose a quadratic inequality that does not hold for $\theta$, but holds for any form in $K$.
Our inequality can be written as $\langle \omega,\rho_{\bm{x}} \rangle\ge 0$, where $\rho_{\bm{x}}\in W_4$ is an associated form of some 4-point array $(x_1,x_2,x_3,x_4)$ and $\omega$ is a fixed unit form in $W_4$.

\begin{wrapfigure}{o}{30mm}
\centering
\vskip-4mm
\includegraphics{mppics/pic-40}
\vskip-0mm
\end{wrapfigure}

Let $N\subset \SSS^2\subset V_4$ be the set of unit vectors $v$ such that $\theta(v)<0$.
The set $N$ does not intersect the four equators $e_1$, $e_2$, $e_3$, $e_4$ parallel to the facets of $\triangle$.
Moreover, since $K\supset K_0$, the set $N$ (up to sign) lies in one of triangles, say $T$;
we can assume that equators $e_1$, $e_2$, $e_3$ are extensions of sides of $T$, and $e_4$ is the remaining equator.

Since $K\supset K_0$, we have $\omega(v)\ge 0$ for any $v\in V_4$.
Let $\sigma$ be a unit 1-form on $V_4$ that vanishes on $e_4$.
By the spectral theorem, $\omega$ and $\sigma^2$ can be diagonalized in a common basis (which does not have to be orthogonal).
Therefore,
\[\langle \omega,\rho_{\bm{x}}\rangle=\rho_{\bm{x}}(u)+\rho_{\bm{x}}(v)+\rho_{\bm{x}}(w)\eqlbl{eq:v_123}\]
for fixed vectors $u,v,w\in V_4$ such that $v$ and $w$ point in the direction of the equator $e_4$ or vanish;
the latter condition follows since $\sigma^2$ is also diagonalized.

The inequalities $\rho_{\bm{x}}(v)\ge 0$ and
$\rho_{\bm{x}}(w)\ge 0$ are of type $(2,1)$;
they follow from the triangle inequality, so we always have that $\rho_{\bm{x}}(v)\ge 0$ and
$\rho_{\bm{x}}(w)\ge 0$.
Moreover, the values $\rho_{\bm{x}}(v)$ and $\rho_{\bm{x}}(w)$ depend only on the sides of the triangle $[x_1x_2x_3]$.

Since
\[
\begin{aligned}
\theta(v)&\ge0,
&
\theta(w)&\ge0,
&
\text{and}&&
\langle\omega,\theta\rangle=\theta(u)+\theta(v)+\theta(w)&<0,
\end{aligned}
\eqlbl{eq:uvw}
\]
we have that $u$ points in a direction of $N$.
Therefore, $\rho_{\bm{x}}(u)\ge0$ is an inequality of negative type $(3,1)$.

Suppose that the array $\bm{x}=(x_1,x_2,x_3,x_4)$ is equipped with a metric such that its associated form is $\theta$.
Let us rescale $u$, $v$, and $w$ so that the inequality $\theta(u)\ge 0$ (which as we know has negative type $(3,1)$)
has $\lambda$-array $(\lambda_1,\lambda_2,\lambda_3,-1)$.
Since $\delta$ is small, \ref{eq:uvw} and \ref{lem:area-bound} imply that
\[\theta(v)+\theta(w)
\le
13\cdot\lambda_1\cdot\lambda_2\cdot\lambda_3\cdot\tilde a^2.\]

For a general array $\bm{x}=(x_1,x_2,x_3,x_4)$, we have
\[\rho_{\bm{x}}(v)=\eps_v\cdot \tilde d_v^2
\quad\text{and}\quad
 \rho_{\bm{x}}(w)=\eps_w\cdot \tilde d_w^2,
\]
where
$\tilde d_v$ and $\tilde d_w$ are distances from a vertex of the model triangle $\tilde\triangle(x_1x_2x_3)_{\EE^2}$ to a point on the opposite side that divides it in a certain ratio and $\eps_v,\eps_w\ge 0$.

In particular, for our choice of $\bm{x}$ it implies $\theta(v)=\eps_v\cdot \tilde d_v^2$ and $\theta(w)=\eps_w\cdot \tilde d_w^2$
Since the diameter of $\tilde\triangle(x_1x_2x_3)$ is at most $\delta$, we have
$\tilde a\z\le \delta\cdot \tilde d_v/2$ and
$\tilde a\z\le \delta\cdot \tilde d_w/2$.
Hence
\[\eps_v+\eps_w\le \tfrac{13}4\cdot \delta^2\cdot\lambda_1\cdot\lambda_2\cdot\lambda_3.\]

Note that for a general 4-point array, $\tilde d_v$ and $\tilde d_w$ do not exceed the maximal side of the triangle $x_1x_2x_3$ --- hence the result.
\qeds

\section{Remarks}

The requirement made in \ref{thm:globalization} that the cone $K$ is closed is necessary;
without it, the globalization holds for the cone described by the inequality
\[|x_1-x_2|^2+|x_1-x_3|^2+|x_2-x_3|^2<4\cdot(|x_1-x_4|^2+|x_2-x_4|^2+|x_3-x_4|^2),\]
assuming that the right-hand side is positive.
To verify this statement, note that the inequality forbids tripods.

It would be super-nice to find a quadratic inequality for $n$-point arrays with globalization that is not implied by the standard globalization theorem.
Some candidates can be found in our earlier paper \cite{lebedeva-petrunin-zolotov}.
Also, it might be possible to prove a version of our globalization theorem
for simply connected spaces, which includes the Cartan--Hadamard theorem.

For general (closed) conditions on semimetrics on $n$-point arrays, globalization holds of course for curvature bounded below by $\kappa\in \RR$ in the sense of Alexandrov,
but it holds for other conditions as well;
the set of semimetrics on $4$-point arrays that are embeddable in $r\cdot\SSS^1$ for any $r>0$ gives an example.
Therefore, understanding globalization phenomena in such general settings is even more interesting.

{\sloppy
\def\emph{\textit}
\printbibliography[heading=bibintoc]
\fussy
}

\end{document}
